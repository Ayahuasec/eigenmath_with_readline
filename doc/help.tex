\section{Function Reference}

\subsection*{abs($x$)}

Returns the absolute value or vector length of $x$.

{\color{blue}
\begin{verbatim}
X = (x,y,z)
abs(X)
\end{verbatim}
}

\noindent
$\left(x^2+y^2+z^2\right)^{1/2}$

\subsection*{adj($m$)}

Returns the adjunct of matrix $m$.
Adjunct is equal to determinant times inverse.

{\color{blue}
\begin{verbatim}
A = ((a,b),(c,d))
adj(A) == det(A) inv(A)
\end{verbatim}
}

\noindent
$1$

\subsection*{and($a,b,\ldots$)}

Returns 1 if all arguments are true (nonzero).
Returns 0 otherwise.

{\color{blue}
\begin{verbatim}
and(1=1,2=2)
\end{verbatim}
}

\noindent
$1$

\subsection*{arccos($x$)}

Returns the arc cosine of $x$.

{\color{blue}
\begin{verbatim}
arccos(1/2)
\end{verbatim}
}

\noindent
$\tfrac{1}{3}\pi$

\subsection*{arccosh($x$)}

Returns the arc hyperbolic cosine of $x$.

\subsection*{arcsin($x$)}

Returns the arc sine of $x$.

{\color{blue}
\begin{verbatim}
arcsin(1/2)
\end{verbatim}
}

\noindent
$\tfrac{1}{6}\pi$

\subsection*{arcsinh($x$)}

Returns the arc hyperbolic sine of $x$.

\subsection*{arctan($y,x$)}

Returns the arc tangent of $y$ over $x$.
If $x$ is omitted then $x=1$ is used.

{\color{blue}
\begin{verbatim}
arctan(1,0)
\end{verbatim}
}

\noindent
$\tfrac{1}{2}\pi$

\subsection*{arctanh($x$)}

Returns the arc hyperbolic tangent of $x$.

\subsection*{arg($z$)}

Returns the angle of complex $z$.

{\color{blue}
\begin{verbatim}
arg(2 - 3i)
\end{verbatim}
}

\noindent
$\arctan(-3,2)$

\subsection*{besselj($x,n$)}

Returns a solution to the Bessel differential equation.

{\color{blue}
\begin{verbatim}
besselj(x,1/2)
\end{verbatim}
}

\noindent
$\displaystyle \frac{2^{1/2}\sin(x)}{\pi^{1/2}\,x^{1/2}}$

\subsection*{binding($s$)}

The result of evaluating a symbol can differ from the symbol's binding.
For example, the result may be expanded.
The {\tt binding} function returns the actual binding of a symbol.

{\color{blue}
\begin{verbatim}
p = quote((x + 1)^2)
p
\end{verbatim}
}

\noindent
$p=x^2+2x+1$

{\color{blue}
\begin{verbatim}
binding(p)
\end{verbatim}
}

\noindent
$(x+1)^2$

\subsection*{binomial($n,k$)}

Returns the coefficient of $x^ky^{n-k}$ in $(x+y)^n$.
Binomial and {\tt choose} are the same function.

{\color{blue}
\begin{verbatim}
binomial(52,5)
\end{verbatim}
}

\noindent
$2598960$

\subsection*{ceiling($x$)}

Returns the smallest integer greater than or equal to $x$.

{\color{blue}
\begin{verbatim}
ceiling(1/2)
\end{verbatim}
}

\noindent
$1$

\subsection*{check($x$)}

If $x$ is true (nonzero) then continue in a script, else stop.
Use {\tt A=B} or {\tt A==B} to test for A equals B.

{\color{blue}
\begin{verbatim}
A = 1
B = 1
check(A=B) -- script stops here if A not equal to B
\end{verbatim}
}

\subsection*{choose($n,k$)}

Returns the number of combinations of $n$ items taken $k$ at a time.
The following example computes the number of poker hands.

{\color{blue}
\begin{verbatim}
choose(52,5)
\end{verbatim}
}

\noindent
$2598960$

\subsection*{circexp($x$)}

Returns expression $x$ with circular and hyperbolic functions
converted to exponentials.

{\color{blue}
\begin{verbatim}
circexp(cos(x) + i sin(x))
\end{verbatim}
}

\noindent
$\exp(ix)$

\subsection*{clear}

Clears all symbol definitions.

\subsection*{clock($z$)}

Returns complex $z$ in polar form with base of negative 1 instead of $e$.

{\color{blue}
\begin{verbatim}
clock(2 - 3i)
\end{verbatim}
}

\noindent
$13^{1/2}\,(-1)^{\arctan(-3,2)/\pi}$

\subsection*{coeff($p,x,n$)}

Returns the coefficient of $x^n$ in polynomial $p$.

{\color{blue}
\begin{verbatim}
p = x^3 + 6x^2 + 12x + 8
coeff(p,x,2)
\end{verbatim}
}

\noindent
$6$

\subsection*{cofactor($m,i,j$)}

Returns a cofactor of matrix $m$.
The cofactor matrix is the transpose of the adjunct of $m$.
This function returns the cofactor component
at row $i$ and column $j$.

{\color{blue}
\begin{verbatim}
A = ((a,b),(c,d))
cofactor(A,1,2) == transpose(adj(A))[1,2]
\end{verbatim}
}

\noindent
$1$

\subsection*{conj($z$)}

Returns the complex conjugate of $z$.

{\color{blue}
\begin{verbatim}
conj(2 - 3i)
\end{verbatim}
}

\noindent
$2 + 3 i$

\subsection*{contract($a,i,j$)}

Returns tensor $a$ summed over indices $i$ and $j$.
If $i$ and $j$ are omitted then 1 and 2 are used.
The expression {\tt contract(m)} computes the trace of matrix $m$.

{\color{blue}
\begin{verbatim}
A = ((a,b),(c,d))
contract(A)
\end{verbatim}
}

\noindent
$a + d$

\subsection*{cos($x$)}

Returns the cosine of $x$.

{\color{blue}
\begin{verbatim}
cos(pi/4)
\end{verbatim}
}

\noindent
$\displaystyle \frac{1}{2^{1/2}}$

\subsection*{cosh($x$)}

Returns the hyperbolic cosine of $x$.

{\color{blue}
\begin{verbatim}
circexp(cosh(x))
\end{verbatim}
}

\noindent
$\tfrac{1}{2}\exp(-x)+\tfrac{1}{2}\exp(x)$


\subsection*{cross($u,v$)}

Returns the cross product of vectors $u$ and $v$.
It is OK to redefine \verb$cross$.
This is the default definition.

{\color{blue}
\begin{verbatim}
cross(u,v) = (u[2] v[3] - u[3] v[2],
              u[3] v[1] - u[1] v[3],
              u[1] v[2] - u[2] v[1])
\end{verbatim}
}

\subsection*{curl($u$)}

Returns the curl of vector $u$.
It is OK to redefine {\tt curl}.
This is the default definition.

{\color{blue}
\begin{verbatim}
curl(u) = (d(u[3],y) - d(u[2],z),
           d(u[1],z) - d(u[3],x),
           d(u[2],x) - d(u[1],y))
\end{verbatim}
}

\subsection*{d($f,x$)}

Returns the partial derivative of $f$ with respect to $x$.

{\color{blue}
\begin{verbatim}
d(x^2,x)
\end{verbatim}
}

\noindent
$2x$

\bigskip
\noindent
Argument $f$ can be a tensor of any rank.
Argument $x$ can be a vector.
When $x$ is a vector the result is the gradient of $f$.

{\color{blue}
\begin{verbatim}
F = (f(),g(),h())
X = (x,y,z)
d(F,X)
\end{verbatim}
}

\noindent
$\displaystyle \begin{bmatrix}
\operatorname{d}(f(),x) & \operatorname{d}(f(),y) &  \operatorname{d}(f(),z)\\
\operatorname{d}(g(),x) & \operatorname{d}(g(),y) &  \operatorname{d}(g(),z)\\
\operatorname{d}(h(),x) & \operatorname{d}(h(),y) &  \operatorname{d}(h(),z)
\end{bmatrix}
$

\bigskip
\noindent
It is OK to use {\tt d} as a variable name.
It will not conflict with function {\tt d}.

\bigskip
\noindent
It is OK to redefine {\tt d} as a different function.
The function {\tt derivative}, a synonym for {\tt d},
can still be used to obtain a partial derivative.

{\color{blue}
\begin{verbatim}
d(x,y) = x - y
derivative(x^2,x)
\end{verbatim}
}

\noindent
$2x$

\subsection*{defint($f,x,a,b$)}

Returns the definite integral of $f$ with respect to $x$
evaluated from $a$ to $b$.
The argument list can be extended for multiple integrals
as shown in the following example.

{\color{blue}
\begin{verbatim}
f = (1 + cos(theta)^2) sin(theta)
defint(f, theta, 0, pi, phi, 0, 2pi) -- integrate over theta then over phi
\end{verbatim}
}

\noindent
$\tfrac{16}{3}\pi$

\subsection*{deg($p,x$)}

Returns the degree of polynomial $p(x)$.

{\color{blue}
\begin{verbatim}
p = (2x + 1)^3
deg(p,x)
\end{verbatim}
}

\noindent
$3$

\subsection*{denominator($x$)}

Returns the denominator of expression $x$.

{\color{blue}
\begin{verbatim}
denominator(a/b)
\end{verbatim}
}

\noindent
$b$

\subsection*{det($m$)}

Returns the determinant of matrix $m$.

{\color{blue}
\begin{verbatim}
A = ((a,b),(c,d))
det(A)
\end{verbatim}
}

\noindent
$a d - b c$

\subsection*{dim($a,n$)}

Returns the dimension of the $n$th index of tensor $a$.
Index numbering starts with 1.

{\color{blue}
\begin{verbatim}
A = ((1,2),(3,4),(5,6))
dim(A,1)
\end{verbatim}
}

\noindent
$3$

\subsection*{div($u$)}

Returns the divergence of vector $u$.
It is OK to redefine {\tt div}.
This is the default definition.

{\color{blue}
\begin{verbatim}
div(u) = d(u[1],x) + d(u[2],y) + d(u[3],z)
\end{verbatim}
}

\subsection*{do($a,b,\ldots$)}

Evaluates each argument from left to right.
Returns the result of the last argument.

{\color{blue}
\begin{verbatim}
do(A=1,B=2,A+B)
\end{verbatim}
}

\noindent
$3$

\subsection*{dot($a,b,\ldots$)}

Returns the dot or matrix product of vectors, matrices, and tensors.

{\color{blue}
\begin{verbatim}
-- solve for X in AX=B
A = ((1,2),(3,4))
B = (5,6)
X = dot(inv(A),B)
X
\end{verbatim}
}

\noindent
$\displaystyle \begin{bmatrix}-4\\ \tfrac{9}{2}\end{bmatrix}$

\subsection*{draw($f,x$)}

Draws a graph of $f(x)$.
Drawing ranges can be set with {\tt xrange} and {\tt yrange}.

\subsection*{e}

Symbol {\tt e} is initialized to the natural number $e$.
It is OK to clear or redefine {\tt e} and use the symbol for something else.

{\color{blue}
\begin{verbatim}
e^x
\end{verbatim}
}

\noindent
$\exp(x)$

\subsection*{eigen($m$)}

Computes eigenvalues and eigenvectors numerically.
Matrix $m$ is required to be both numerical and symmetric.
Eigenvectors are returned in Q and eigenvalues are returned in D.
Each row of Q is an eigenvector.
Each diagonal element of D is an eigenvalue.

{\color{blue}
\begin{verbatim}
A = ((1,2),(2,1))
eigen(A)
dot(transpose(Q),D,Q)
\end{verbatim}
}

\noindent
$\displaystyle \begin{bmatrix}
1.0 & 2.0\\
2.0 & 1.0
\end{bmatrix}
$

\subsection*{erf($x$)}

Error function of $x$.

\subsection*{erfc($x$)}

Complementary error function of $x$.

\subsection*{eval($f,x,a$)}

Returns $f$ evaluated at $x$ equals $a$.

{\color{blue}
\begin{verbatim}
eval(x^2 + 3,x,0)
\end{verbatim}
}

\noindent
$3$

\subsection*{exp($x$)}

Returns the exponential of $x$.

{\color{blue}
\begin{verbatim}
exp(i pi)
\end{verbatim}
}

\noindent
$-1$

\subsection*{expand($r,x$)}

Returns the partial fraction expansion of the ratio of polynomials $r$ in $x$.

{\color{blue}
\begin{verbatim}
p = (x + 1)^2
q = (x + 2)^2
expand(p/q,x)
\end{verbatim}
}

\noindent
$\displaystyle -\frac{2}{x+2}+\frac{1}{x^2+4x+4}+1$

\subsection*{expcos($z$)}

Returns the cosine of $z$ in exponential form.

{\color{blue}
\begin{verbatim}
expcos(z)
\end{verbatim}
}

\noindent
$\displaystyle \tfrac{1}{2}\exp(iz)+\tfrac{1}{2}\exp(-iz)$

\subsection*{expcosh($z$)}

Returns the hyperbolic cosine of $z$ in exponential form.

{\color{blue}
\begin{verbatim}
expcosh(z)
\end{verbatim}
}

\noindent
$\displaystyle \tfrac{1}{2}\exp(-z)+\tfrac{1}{2}\exp(z)$

\subsection*{expsin($z$)}

Returns the sine of $z$ in exponential form.

{\color{blue}
\begin{verbatim}
expsin(z)
\end{verbatim}
}

\noindent
$\displaystyle -\tfrac{1}{2}i\exp(iz)+\tfrac{1}{2}i\exp(-iz)$

\subsection*{expsinh($z$)}

Returns the hyperbolic sine of $z$ in exponential form.

{\color{blue}
\begin{verbatim}
expsinh(z)
\end{verbatim}
}

\noindent
$\displaystyle -\tfrac{1}{2}\exp(-z)+\tfrac{1}{2}\exp(z)$

\subsection*{exptan($z$)}

Returns the tangent of $z$ in exponential form.

{\color{blue}
\begin{verbatim}
exptan(z)
\end{verbatim}
}

\noindent
$\displaystyle \frac{i}{\exp(2iz)+1}-\frac{i\exp(2iz)}{\exp(2iz)+1}$

\subsection*{exptanh($z$)}

Returns the hyperbolic tangent of $z$ in exponential form.

{\color{blue}
\begin{verbatim}
exptanh(z)
\end{verbatim}
}

\noindent
$\displaystyle -\frac{1}{\exp(2z)+1}+\frac{\exp(2z)}{\exp(2z)+1}$

\subsection*{factor($n$)}

Factors integer $n$.

{\color{blue}
\begin{verbatim}
factor(10!)
\end{verbatim}
}

\noindent
$\displaystyle 2^8\times 3^4\times 5^2\times 7^1$

\subsection*{factor($p,x$)}

Factors polynomial $p(x)$.
The argument list can be extended for multivariate polynomials.

{\color{blue}
\begin{verbatim}
p = 2x + x y + y + 2
factor(p,x,y)
\end{verbatim}
}

\noindent
$\displaystyle (x+1)(y+2)$

\bigskip
\noindent
Note: Factor returns an unexpanded expression.
If the result is assigned to a symbol, evaluating
the symbol will expand the result.
Use {\tt binding} to retrieve the unexpanded expression.

{\color{blue}
\begin{verbatim}
q = factor(p,x)
binding(q)
\end{verbatim}
}

\noindent
$\displaystyle (x+1)(y+2)$

\subsection*{factorial($n$)}

Returns the factorial of $n$.
The expression {\tt n!} can also be used.

{\color{blue}
\begin{verbatim}
100!
\end{verbatim}
}

\noindent
$93326215443944152681699238856266700490715968264381621468592963895217599993229915$\\
$608941463976156518286253697920827223758251185210916864000000000000000000000000$

\subsection*{filter($f,a,b,\ldots$)}

Returns $f$ excluding any terms containing $a$, $b$, etc.

{\color{blue}
\begin{verbatim}
p = x^2 + 3x + 2
filter(p,x^2)
\end{verbatim}
}

\noindent
$3x+2$

\subsection*{float($x$)}

Returns expression $x$ with rational numbers and integers converted to
floating point values.
The symbol {\tt pi} and the natural number are also converted.

{\color{blue}
\begin{verbatim}
float(212^17)
\end{verbatim}
}

\noindent
$\displaystyle 3.52947\times 10^{39}$

\subsection*{floor($x$)}

Returns the largest integer less than or equal to $x$.

\par
{\color{blue}
\begin{verbatim}
floor(1/2)
\end{verbatim}
}

\noindent
$0$

\subsection*{for($i,j,k,a,b,\ldots$)}

For $i$ equals $j$ through $k$ evaluate $a$, $b$, etc.
The original value of symbol $i$ is restored after {\tt for} completes.

{\color{blue}
\begin{verbatim}
for(k,1,3,A=k,print(A))
\end{verbatim}
}

\noindent
$A=1$\\
$A=2$\\
$A=3$

\subsection*{gcd($a,b,\ldots$)}

Returns the greatest common divisor of expressions.

{\color{blue}
\begin{verbatim}
gcd(x,x y)
\end{verbatim}
}

\noindent
$x$

\subsection*{hermite($x,n$)}

Returns the $n$th Hermite polynomial in $x$.

{\color{blue}
\begin{verbatim}
hermite(x,3)
\end{verbatim}
}

\noindent
$\displaystyle 8x^3-12x$


\subsection*{hilbert($n$)}

Returns an $n$ by $n$ Hilbert matrix.

{\color{blue}
\begin{verbatim}
hilbert(3)
\end{verbatim}
}

\noindent
$\displaystyle
\begin{bmatrix}
1 & \tfrac{1}{2} & \tfrac{1}{3}\\ \\
\tfrac{1}{2} & \tfrac{1}{3} & \tfrac{1}{4}\\ \\
\tfrac{1}{3} & \tfrac{1}{4} & \tfrac{1}{5}
\end{bmatrix}
$

\iffalse

<p style="font-family:courier;font-size:20pt;font-weight:bold">
<a name="i">i</a>
<p>
Symbol <tt>i</tt> is initialized to the imaginary unit (&minus;1)<sup>1/2</sup>.
It is OK to clear or redefine <tt>i</tt> and use the symbol for something else.
<pre style="color:blue">
exp(i pi)
</pre>
<pre>
-1
</pre>

<p style="font-family:courier;font-size:20pt;font-weight:bold">
<a name="imag">imag(<i>z</i>)</a>
<p>
Returns the imaginary part of complex <i>z</i>.
<pre style="color:blue">
imag(2 - 3i)
</pre>
<pre>
-3
</pre>

<p style="font-family:courier;font-size:20pt;font-weight:bold">
<a name="inner">inner(<i>a,b,...</i>)</a>
<p>
Returns the inner product of tensors.
Inner and <tt>dot</tt> are the same function.
<pre style="color:blue">
A = ((a,b),(c,d))
B = (x,y)
inner(A,B)
</pre>
<pre>
a x + b y

c x + d y
</pre>
<p>
Note: Inner product is equivalent to an outer product followed by contraction.
<pre style="color:blue">
contract(outer(A,B),2,3)
</pre>
<pre>
a x + b y

c x + d y
</pre>

<p style="font-family:courier;font-size:20pt;font-weight:bold">
<a name="integral">integral(<i>f,x</i>)</a>
<p>
Returns the integral of <i>f</i> with respect to <i>x</i>.
<pre style="color:blue">
integral(x^2,x)
</pre>
<pre>
 1   3
--- x
 3
</pre>

<p style="font-family:courier;font-size:20pt;font-weight:bold">
<a name="inv">inv(<i>m</i>)</a>
<p>
Returns the inverse of matrix <i>m</i>.
<pre style="color:blue">
A = ((1,2),(3,4))
inv(A)
</pre>
<pre>
 -2       1


  3        1
 ---    - ---
  2        2
</pre>

<p style="font-family:courier;font-size:20pt;font-weight:bold">
<a name="isprime">isprime(<i>n</i>)</a>
<p>
Returns 1 if <i>n</i> is a prime number. Returns zero otherwise.
<pre style="color:blue">
isprime(2^31 - 1)
</pre>
<pre>
1
</pre>

<p style="font-family:courier;font-size:20pt;font-weight:bold">
<a name="j">j</a>
<p>
Set <tt>j=sqrt(-1)</tt> to use <tt>j</tt> for the imaginary unit instead of <tt>i</tt>.
<pre style="color:blue">
j = sqrt(-1)
1/sqrt(-1)
</pre>
<pre>
-j
</pre>

<p style="font-family:courier;font-size:20pt;font-weight:bold">
<a name="laguerre">laguerre(<i>x,n,a</i>)</a>
<p>
Returns the <i>n</i>th Laguerre polynomial in <i>x</i>.
If argument <i>a</i> is omitted then zero is used.
<pre style="color:blue">
laguerre(x,3)
</pre>
<pre>
   1   3    3   2
- --- x  + --- x  - 3 x + 1
   6        2
</pre>

<p style="font-family:courier;font-size:20pt;font-weight:bold">
<a name="last">last</a>
<p>
The result of the previous calculation is stored in <tt>last</tt>.
<pre style="color:blue">
212^17
</pre>
<pre>
3529471145760275132301897342055866171392
</pre>
<pre style="color:blue">
last^(1/17)
</pre>
<pre>
212
</pre>
<p>
Note: Symbol <tt>last</tt> is an implied argument when a function has no
argument list.
<pre style="color:blue">
212^17
</pre>
<pre>
3529471145760275132301897342055866171392
</pre>
<pre style="color:blue">
float
</pre>
<pre>
          39
3.52947 10
</pre>

<p style="font-family:courier;font-size:20pt;font-weight:bold">
<a name="lcm">lcm(<i>a,b,...</i>)</a>
<p>
Returns the least common multiple of expressions.
<pre style="color:blue">
lcm(x,x y)
</pre>
<pre>
x y
</pre>

<p style="font-family:courier;font-size:20pt;font-weight:bold">
<a name="leading">leading(<i>p,x</i>)</a>
<p>
Returns the leading coefficient of polynomial <i>p</i>(<i>x</i>).
<pre style="color:blue">
leading(3x^2 + 1,x)
</pre>
<pre>
3
</pre>

<p style="font-family:courier;font-size:20pt;font-weight:bold">
<a name="legendre">legendre(<i>x,n,m</i>)</a>
<p>
Returns the <i>n</i>th Legendre polynomial in <i>x</i>.
If <i>m</i> is omitted then zero is used.
<pre style="color:blue">
legendre(x,3)
</pre>
<pre>
 5   3    3
--- x  - --- x
 2        2
</pre>

<p style="font-family:courier;font-size:20pt;font-weight:bold">
<a name="lisp">lisp(<i>x</i>)</a>
<p>
Evaluates expression <i>x</i> and returns the result as a
string in prefix notation.
Useful for debugging scripts.
<pre style="color:blue">
lisp(x^2 + 1)
</pre>
<pre>
(+ (^ x 2) 1)
</pre>

<p style="font-family:courier;font-size:20pt;font-weight:bold">
<a name="log">log(<i>x</i>)</a>
<p>
Returns the natural logarithm of <i>x</i>.
<pre style="color:blue">
log(x^y)
</pre>
<pre>
y log(x)
</pre>

<p style="font-family:courier;font-size:20pt;font-weight:bold">
<a name="mag">mag(<i>z</i>)</a>
<p>
Returns the magnitude of complex <i>z</i>.
Mag treats undefined symbols as real while <tt>abs</tt> does not.
<pre style="color:blue">
mag(x + i y)
</pre>
<pre>
         1/2
  2    2
(x  + y )
</pre>

<p style="font-family:courier;font-size:20pt;font-weight:bold">
<a name="mod">mod(<i>a,b</i>)</a>
<p>
Returns the remainder of integer <i>a</i> divided by integer <i>b</i>.
<pre style="color:blue">
mod(10,7)
</pre>
<pre>
3
</pre>

<p style="font-family:courier;font-size:20pt;font-weight:bold">
<a name="not">not(<i>x</i>)</a>
<p>
Returns 0 if <i>x</i> is true (nonzero).
Returns 1 otherwise.
<pre style="color:blue">
not(1=1)
</pre>
<pre>
0
</pre>

<p style="font-family:courier;font-size:20pt;font-weight:bold">
<a name="nroots">nroots(<i>p,x</i>)</a>
<p>
Returns all roots, both real and complex,
of polynomial <i>p</i>(<i>x</i>).
The roots are computed numerically.
The coefficients of <i>p</i> can be real or complex.

<p style="font-family:courier;font-size:20pt;font-weight:bold">
<a name="number">number(<i>x</i>)</a>
<p>
Returns 1 if <i>x</i> is a rational or floating point number.
Returns 0 otherwise.
<pre style="color:blue">
number(1/2)
</pre>
<pre>
1
</pre>

<p style="font-family:courier;font-size:20pt;font-weight:bold">
<a name="numerator">numerator(<i>x</i>)</a>
<p>
Returns the numerator of expression <i>x</i>.
<pre style="color:blue">
numerator(a/b)
</pre>
<pre>
a
</pre>

<p style="font-family:courier;font-size:20pt;font-weight:bold">
<a name="or">or(<i>a,b,...</i>)</a>
<p>
Returns 1 if at least one argument is true (nonzero).
Returns 0 otherwise.
<pre style="color:blue">
or(1=1,2=2)
</pre>
<pre>
1
</pre>

<p style="font-family:courier;font-size:20pt;font-weight:bold">
<a name="outer">outer(<i>a,b,...</i>)</a>
<p>
Returns the outer product of tensors.
Also known as the tensor product.
<pre style="color:blue">
A = (a,b,c)
B = (x,y,z)
outer(A,B)
</pre>
<pre>
a x   a y   a z

b x   b y   b z

c x   c y   c z
</pre>

<p style="font-family:courier;font-size:20pt;font-weight:bold">
<a name="pi">pi</a>
<p>
Symbol for &#960;.
<pre style="color:blue">
exp(i pi)
</pre>
<pre>
-1
</pre>

<p style="font-family:courier;font-size:20pt;font-weight:bold">
<a name="polar">polar(<i>z</i>)</a>
<p>
Returns complex <i>z</i> in polar form.
<pre style="color:blue">
polar(x - i y)
</pre>
<pre>
         1/2
  2    2
(x  + y )    exp(i arctan(-y,x))
</pre>

<p style="font-family:courier;font-size:20pt;font-weight:bold">
<a name="power">power</a>
<p>
Use <tt>^</tt> to raise something to a power.
Use parentheses for negative powers.
<pre style="color:blue">
x^(-2)
</pre>
<pre>
 1
----
  2
 x
</pre>

<p style="font-family:courier;font-size:20pt;font-weight:bold">
<a name="prime">prime(<i>n</i>)</a>
<p>
Returns the <i>n</i>th prime number.
The domain of <i>n</i> is 1 to 10000.
<pre style="color:blue">
prime(100)
</pre>
<pre>
541
</pre>

<p style="font-family:courier;font-size:20pt;font-weight:bold">
<a name="print">print(<i>a,b,...</i>)</a>
<p>
Evaluate expressions and print the results.
Useful for printing from inside a <tt>for</tt> loop.
<pre style="color:blue">
for(j,1,3,print(j))
</pre>
<pre>
j = 1
j = 2
j = 3
</pre>

<p style="font-family:courier;font-size:20pt;font-weight:bold">
<a name="product">product(<i>i,j,k,f</i>)</a>
<p>
For <i>i</i> equals <i>j</i> through <i>k</i> evaluate <i>f</i>.
Returns the product of all <i>f</i>.
<pre style="color:blue">
product(j,1,3,x + j)
</pre>
<pre>
 3      2
x  + 6 x  + 11 x + 6
</pre>

<p style="font-family:courier;font-size:20pt;font-weight:bold">
<a name="quote">quote(<i>x</i>)</a>
<p>
Returns expression <i>x</i> without evaluating it first.
<pre style="color:blue">
p = quote((x + 1)^2)
binding(p)
</pre>
<pre>
       2
(x + 1)
</pre>
<pre style="color:blue">
p = quote(p) -- clear symbol p
binding(p)
</pre>
<pre>
p
</pre>

<p style="font-family:courier;font-size:20pt;font-weight:bold">
<a name="quotient">quotient(<i>p,q,x</i>)</a>
<p>
Returns the quotient of polynomial <i>p</i>(<i>x</i>) over <i>q</i>(<i>x</i>).
<pre style="color:blue">
p = x^2 + 1
q = x + 3
quotient(p,q,x)
</pre>
<pre>
x - 3
</pre>
<pre style="color:blue">
p - q quotient(p,q,x) -- remainder of p/q
</pre>
<pre>
10
</pre>

<p style="font-family:courier;font-size:20pt;font-weight:bold">
<a name="rank">rank(<i>a</i>)</a>
<p>
Returns the number of indices that tensor <i>a</i> has.
<pre style="color:blue">
A = ((1,0),(0,1))
rank(outer(A,A,A))
</pre>
<pre>
6
</pre>

<p style="font-family:courier;font-size:20pt;font-weight:bold">
<a name="rationalize">rationalize(<i>x</i>)</a>
<p>
Returns expression <i>x</i> with everything over a common denominator.
<pre style="color:blue">
rationalize(1/a + 1/b + 1/2)
</pre>
<pre>
 2 a + a b + 2 b
-----------------
      2 a b
</pre>
<p>
Note: Rationalize returns an unexpanded expression.
If the result is assigned to a symbol,
evaluating the symbol will expand the result.
Use <tt>binding</tt> to retrieve the unexpanded expression.
<pre style="color:blue">
f = rationalize(1/a + 1/b + 1/2)
binding(f)
</pre>
<pre>
 2 a + a b + 2 b
-----------------
      2 a b
</pre>

<p style="font-family:courier;font-size:20pt;font-weight:bold">
<a name="real">real(<i>z</i>)</a>
<p>
Returns the real part of complex <i>z</i>.
<pre style="color:blue">
real(2 - 3i)
</pre>
<pre>
2
</pre>

<p style="font-family:courier;font-size:20pt;font-weight:bold">
<a name="rect">rect(<i>z</i>)</a>
<p>
Returns complex <i>z</i> in rectangular form.
<pre style="color:blue">
rect(exp(i x))
</pre>
<pre>
cos(x) + i sin(x)
</pre>

<p style="font-family:courier;font-size:20pt;font-weight:bold">
<a name="roots">roots(<i>p,x</i>)</a>
<p>
Returns the values of <i>x</i> such that polynomial
<i>p</i>(<i>x</i>) equals zero.
The polynomial should be factorable over integers.
Returns a vector for multiple roots.
<pre style="color:blue">
roots(x^2 + 3x + 2,x)
</pre>
<pre>
-2

-1
</pre>

<p style="font-family:courier;font-size:20pt;font-weight:bold">
<a name="run">run(<i>file</i>)</a>
<p>
Run script <i>file</i>.
Useful for importing function libraries.
<pre style="color:blue">
run("Downloads/EVA.txt")
</pre>
<p>
<i>file</i> must be in the Downloads folder due to security requirements for apps distributed on the Mac App Store.

<p style="font-family:courier;font-size:20pt;font-weight:bold">
<a name="simplify">simplify(<i>x</i>)</a>
<p>
Returns expression <i>x</i> in a simpler form.
<pre style="color:blue">
simplify(sin(x)^2 + cos(x)^2)
</pre>
<pre>
1
</pre>

<p style="font-family:courier;font-size:20pt;font-weight:bold">
<a name="sin">sin(<i>x</i>)</a>
<p>
Returns the sine of <i>x</i>.
<pre style="color:blue">
sin(pi/4)
</pre>
<pre>
  1
------
  1/2
 2
</pre>
<pre style="color:blue">
sin(arctan(y,x))
</pre>
<pre>
      y
--------------
          1/2
   2    2
 (x  + y )
</pre>

<p style="font-family:courier;font-size:20pt;font-weight:bold">
<a name="sinh">sinh(<i>x</i>)</a>
<p>
Returns the hyperbolic sine of <i>x</i>.
<pre style="color:blue">
circexp(sinh(x))
</pre>
<pre>
   1             1
- --- exp(-x) + --- exp(x)
   2             2
</pre>

<p style="font-family:courier;font-size:20pt;font-weight:bold">
<a name="sqrt">sqrt(<i>x</i>)</a>
<p>
Returns the square root of <i>x</i>.
<pre style="color:blue">
sqrt(10!)
</pre>
<pre>
     1/2
720 7
</pre>

<p style="font-family:courier;font-size:20pt;font-weight:bold">
<a name="status">status</a>
<p>
Prints memory statistics.
<pre style="color:blue">
status
</pre>
<pre>
block_count 1
free_count 99258
gc_count 1
bignum_count 370
string_count 0
tensor_count 5
</pre>

<p style="font-family:courier;font-size:20pt;font-weight:bold">
<a name="stop">stop</a>
<p>
In a script, it does what it says.

<p style="font-family:courier;font-size:20pt;font-weight:bold">
<a name="string">string(<i>x</i>)</a>
<p>
Evaluates expression <i>x</i> and returns the result as a string.
Useful for testing scripts.
<pre style="color:blue">
string((x + 1)^2) == "x^2 + 2 x + 1"
</pre>
<pre>
1
</pre>

<p style="font-family:courier;font-size:20pt;font-weight:bold">
<a name="subst">subst(<i>a,b,c</i>)</a>
<p>
Substitutes <i>a</i> for <i>b</i> in <i>c</i> and returns the result.
<pre style="color:blue">
subst(x,y,y^2)
</pre>
<pre>
 2
x
</pre>

<p style="font-family:courier;font-size:20pt;font-weight:bold">
<a name="sum">sum(<i>i,j,k,f</i>)</a>
<p>
For <i>i</i> equals <i>j</i> through <i>k</i> evaluate <i>f</i>.
Returns the sum of all <i>f</i>.
<pre style="color:blue">
sum(j,1,5,x^j)
</pre>
<pre>
 5    4    3    2
x  + x  + x  + x  + x
</pre>

<p style="font-family:courier;font-size:20pt;font-weight:bold">
<a name="tan">tan(<i>x</i>)</a>
<p>
Returns the tangent of <i>x</i>.
<pre style="color:blue">
simplify(tan(x) - sin(x)/cos(x))
</pre>
<pre>
0
</pre>

<p style="font-family:courier;font-size:20pt;font-weight:bold">
<a name="tanh">tanh(<i>x</i>)</a>
<p>
Returns the hyperbolic tangent of <i>x</i>.
<pre style="color:blue">
circexp(tanh(x))
</pre>
<pre>
        1             exp(2 x)
- -------------- + --------------
   exp(2 x) + 1     exp(2 x) + 1
</pre>

<p style="font-family:courier;font-size:20pt;font-weight:bold">
<a name="taylor">taylor(<i>f,x,n,a</i>)</a>
<p>
Returns the Taylor expansion of <i>f</i>(<i>x</i>) near <i>x</i> equals <i>a</i>.
If argument <i>a</i> is omitted then zero is used.
Argument <i>n</i> is the degree of the expansion.
<pre style="color:blue">
taylor(sin(x),x,5)
</pre>
<pre>
  1    5    1   3
----- x  - --- x  + x
 120        6
</pre>

<p style="font-family:courier;font-size:20pt;font-weight:bold">
<a name="test">test(<i>a,b,c,d,...</i>)</a>
<p>
If argument <i>a</i> is true (nonzero) then <i>b</i> is returned,
else if <i>c</i> is true then <i>d</i> is returned, etc.
If the number of arguments is odd then the last argument is returned
if all else fails.
Use A=B or A==B to test for A equals B.
<pre style="color:blue">
A = 1
B = 1
test(A=B,"yes","no")
</pre>
<pre>
yes
</pre>

<p style="font-family:courier;font-size:20pt;font-weight:bold">
<a name="trace">trace</a>
<p>
Set <tt>trace=1</tt> in a script to print the script as it is evaluated.
Useful for debugging.
<pre style="color:blue">
trace = 1
</pre>
<p>
Note: The <tt>contract</tt> function is used to obtain the trace of a matrix.

<p style="font-family:courier;font-size:20pt;font-weight:bold">
<a name="transpose">transpose(<i>a,i,j</i>)</a>
<p>
Returns the transpose of tensor <i>a</i> with respect to indices <i>i</i> and <i>j</i>.
If <i>i</i> and <i>j</i> are omitted then 1 and 2 are used.
Hence a matrix can be transposed with a single argument.
<pre style="color:blue">
A = ((a,b),(c,d))
transpose(A)
</pre>
<pre>
a   c

b   d
</pre>

<p style="font-family:courier;font-size:20pt;font-weight:bold">
<a name="tty">tty</a>
<p>
Set <tt>tty=1</tt> to print results in a flat format.
<pre style="color:blue">
tty = 1
(x + 1/2)^2
</pre>
<pre>
x^2 + x + 1/4
</pre>

<p style="font-family:courier;font-size:20pt;font-weight:bold">
<a name="unit">unit(<i>n</i>)</a>
<p>
Returns an <i>n</i> by <i>n</i> identity matrix.
<pre style="color:blue">
unit(3)
</pre>
<pre>
1   0   0

0   1   0

0   0   1
</pre>

<p style="font-family:courier;font-size:20pt;font-weight:bold">
<a name="zero">zero(<i>i,j,...</i>)</a>
<p>
Returns a null tensor with dimensions <i>i</i>, <i>j</i>, etc.
Useful for creating a tensor and then setting the component values.
<pre style="color:blue">
A = zero(3,3)
for(k,1,3,A[k,k]=k)
A
</pre>
<pre>
    1   0   0

A = 0   2   0

    0   0   3
</pre>

\fi
