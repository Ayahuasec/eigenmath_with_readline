\section{Function Reference}

\subsection*{abs($x$)}

Returns the absolute value or vector length of $x$.

{\color{blue}
\begin{verbatim}
X = (x,y,z)
abs(X)
\end{verbatim}
}

\noindent
$\left(x^2+y^2+z^2\right)^{1/2}$

\subsection*{adj($m$)}

Returns the adjunct of matrix $m$.
Adjunct is equal to determinant times inverse.

{\color{blue}
\begin{verbatim}
A = ((a,b),(c,d))
adj(A) == det(A) inv(A)
\end{verbatim}
}

\noindent
$1$

\subsection*{and($a,b,\ldots$)}

Returns 1 if all arguments are true (nonzero).
Returns 0 otherwise.

{\color{blue}
\begin{verbatim}
and(1=1,2=2)
\end{verbatim}
}

\noindent
$1$

\subsection*{arccos($x$)}

Returns the arc cosine of $x$.

{\color{blue}
\begin{verbatim}
arccos(1/2)
\end{verbatim}
}

\noindent
$\tfrac{1}{3}\pi$

\subsection*{arccosh($x$)}

Returns the arc hyperbolic cosine of $x$.

\subsection*{arcsin($x$)}

Returns the arc sine of $x$.

{\color{blue}
\begin{verbatim}
arcsin(1/2)
\end{verbatim}
}

\noindent
$\tfrac{1}{6}\pi$

\subsection*{arcsinh($x$)}

Returns the arc hyperbolic sine of $x$.

\subsection*{arctan($y,x$)}

Returns the arc tangent of $y$ over $x$.
If $x$ is omitted then $x=1$ is used.

{\color{blue}
\begin{verbatim}
arctan(1,0)
\end{verbatim}
}

\noindent
$\tfrac{1}{2}\pi$

\subsection*{arctanh($x$)}

Returns the arc hyperbolic tangent of $x$.

\subsection*{arg($z$)}

Returns the angle of complex $z$.

{\color{blue}
\begin{verbatim}
arg(2 - 3i)
\end{verbatim}
}

\noindent
$\arctan(-3,2)$

\subsection*{binding($s$)}

The result of evaluating a symbol can differ from the symbol's binding.
For example, the result may be expanded.
The {\tt binding} function returns the actual binding of a symbol.

{\color{blue}
\begin{verbatim}
p = quote((x + 1)^2)
p
\end{verbatim}
}

\noindent
$p=x^2+2x+1$

{\color{blue}
\begin{verbatim}
binding(p)
\end{verbatim}
}

\noindent
$(x+1)^2$

\subsection*{ceiling($x$)}

Returns the smallest integer greater than or equal to $x$.

{\color{blue}
\begin{verbatim}
ceiling(1/2)
\end{verbatim}
}

\noindent
$1$

\subsection*{check($x$)}

If $x$ is true (nonzero) then continue, else stop.
Expression $x$ can include the relational operators
\verb$=$,
\verb$==$,
\verb$<$,
\verb$<=$,
\verb$>$,
\verb$>=$.
Use the
\verb$not$
function to test for inequality.

{\color{blue}
\begin{verbatim}
A = 1
B = 1
check(A=B) -- stop here if A not equal to B
\end{verbatim}
}

\subsection*{circexp($x$)}

Returns expression $x$ with circular and hyperbolic functions
converted to exponentials.

{\color{blue}
\begin{verbatim}
circexp(cos(x) + i sin(x))
\end{verbatim}
}

\noindent
$\exp(ix)$

\subsection*{clear}

Clears all symbol definitions.

\subsection*{clock($z$)}

Returns complex $z$ in polar form with base of negative 1 instead of $e$.

{\color{blue}
\begin{verbatim}
clock(2 - 3i)
\end{verbatim}
}

\noindent
$13^{1/2}\,(-1)^{\arctan(-3,2)/\pi}$

\subsection*{coeff($m,i,j$)}

Returns the cofactor of matrix $m$ for row $i$ and column $j$.

{\color{blue}
\begin{verbatim}
A = ((a,b),(c,d))
cofactor(A,1,2) == adj(A)[2,1]
\end{verbatim}
}

\noindent
1

\subsection*{conj($z$)}

Returns the complex conjugate of $z$.

{\color{blue}
\begin{verbatim}
conj(2 - 3i)
\end{verbatim}
}

\noindent
$2 + 3 i$

\subsection*{contract($a,i,j$)}

Returns tensor $a$ summed over indices $i$ and $j$.
If $i$ and $j$ are omitted then 1 and 2 are used.
The expression {\tt contract(m)} computes the trace of matrix $m$.

{\color{blue}
\begin{verbatim}
A = ((a,b),(c,d))
contract(A)
\end{verbatim}
}

\noindent
$a + d$

\subsection*{cos($x$)}

Returns the cosine of $x$.

{\color{blue}
\begin{verbatim}
cos(pi/4)
\end{verbatim}
}

\noindent
$\displaystyle \frac{1}{2^{1/2}}$

\subsection*{cosh($x$)}

Returns the hyperbolic cosine of $x$.

{\color{blue}
\begin{verbatim}
circexp(cosh(x))
\end{verbatim}
}

\noindent
$\tfrac{1}{2}\exp(-x)+\tfrac{1}{2}\exp(x)$

\subsection*{d($f,x$)}

Returns the partial derivative of $f$ with respect to $x$.

{\color{blue}
\begin{verbatim}
d(x^2,x)
\end{verbatim}
}

\noindent
$2x$

\bigskip
\noindent
Argument $f$ can be a tensor of any rank.
Argument $x$ can be a vector.
When $x$ is a vector the result is the gradient of $f$.

{\color{blue}
\begin{verbatim}
F = (f(),g(),h())
X = (x,y,z)
d(F,X)
\end{verbatim}
}

\noindent
$\displaystyle \begin{bmatrix}
\operatorname{d}(f(),x) & \operatorname{d}(f(),y) &  \operatorname{d}(f(),z)\\
\operatorname{d}(g(),x) & \operatorname{d}(g(),y) &  \operatorname{d}(g(),z)\\
\operatorname{d}(h(),x) & \operatorname{d}(h(),y) &  \operatorname{d}(h(),z)
\end{bmatrix}
$

\bigskip
\noindent
It is OK to use {\tt d} as a variable name.
It will not conflict with function {\tt d}.

\bigskip
\noindent
It is OK to redefine {\tt d} as a different function.
The function {\tt derivative}, a synonym for {\tt d},
can still be used to obtain a partial derivative.

\subsection*{defint($f,x,a,b$)}

Returns the definite integral of $f$ with respect to $x$
evaluated from $a$ to $b$.
The argument list can be extended for multiple integrals
as shown in the following example.

{\color{blue}
\begin{verbatim}
f = (1 + cos(theta)^2) sin(theta)
defint(f, theta, 0, pi, phi, 0, 2pi) -- integrate over theta then over phi
\end{verbatim}
}

\noindent
$\tfrac{16}{3}\pi$

\subsection*{denominator($x$)}

Returns the denominator of expression $x$.

{\color{blue}
\begin{verbatim}
denominator(a/b)
\end{verbatim}
}

\noindent
$b$

\subsection*{det($m$)}

Returns the determinant of matrix $m$.

{\color{blue}
\begin{verbatim}
A = ((a,b),(c,d))
det(A)
\end{verbatim}
}

\noindent
$a d - b c$

\subsection*{dim($a,n$)}

Returns the dimension of the $n$th index of tensor $a$.
Index numbering starts with 1.

{\color{blue}
\begin{verbatim}
A = ((1,2),(3,4),(5,6))
dim(A,1)
\end{verbatim}
}

\noindent
$3$

\subsection*{do($a,b,\ldots$)}

Evaluates each argument from left to right.
Returns the result of the final argument.

{\color{blue}
\begin{verbatim}
do(A=1,B=2,A+B)
\end{verbatim}
}

\noindent
$3$

\subsection*{dot($a,b,\ldots$)}

Returns the dot product of vectors, matrices, and tensors.
Also known as the matrix product.

{\color{blue}
\begin{verbatim}
-- solve for X in AX=B
A = ((1,2),(3,4))
B = (5,6)
X = dot(inv(A),B)
X
\end{verbatim}
}

\noindent
$\displaystyle \begin{bmatrix}-4\\ \tfrac{9}{2}\end{bmatrix}$

\subsection*{draw($f,x$)}

Draws a graph of $f(x)$.
Drawing ranges can be set with {\tt xrange} and {\tt yrange}.

{\color{blue}
\begin{verbatim}
xrange = (0,1)
yrange = (0,1)
draw(x^2,x)
\end{verbatim}
}

\subsection*{eval($f,x,a$)}

Returns expression $f$ evaluated at $x$ equals $a$.
The argument list can be extended for multivariate expressions.
For example,
\verb$eval(f,x,a,y,b)$
is equivalent to
\verb$eval(eval(f,x,a),y,b)$.

{\color{blue}
\begin{verbatim}
eval(x + y,x,a,y,b)
\end{verbatim}
}

\noindent
$a+b$

\subsection*{exp($x$)}

Returns the exponential of $x$.

{\color{blue}
\begin{verbatim}
exp(i pi)
\end{verbatim}
}

\noindent
$-1$

\subsection*{expcos($z$)}

Returns the cosine of $z$ in exponential form.

{\color{blue}
\begin{verbatim}
expcos(z)
\end{verbatim}
}

\noindent
$\displaystyle \tfrac{1}{2}\exp(iz)+\tfrac{1}{2}\exp(-iz)$

\subsection*{expcosh($z$)}

Returns the hyperbolic cosine of $z$ in exponential form.

{\color{blue}
\begin{verbatim}
expcosh(z)
\end{verbatim}
}

\noindent
$\displaystyle \tfrac{1}{2}\exp(-z)+\tfrac{1}{2}\exp(z)$

\subsection*{expsin($z$)}

Returns the sine of $z$ in exponential form.

{\color{blue}
\begin{verbatim}
expsin(z)
\end{verbatim}
}

\noindent
$\displaystyle -\tfrac{1}{2}i\exp(iz)+\tfrac{1}{2}i\exp(-iz)$

\subsection*{expsinh($z$)}

Returns the hyperbolic sine of $z$ in exponential form.

{\color{blue}
\begin{verbatim}
expsinh(z)
\end{verbatim}
}

\noindent
$\displaystyle -\tfrac{1}{2}\exp(-z)+\tfrac{1}{2}\exp(z)$

\subsection*{exptan($z$)}

Returns the tangent of $z$ in exponential form.

{\color{blue}
\begin{verbatim}
exptan(z)
\end{verbatim}
}

\noindent
$\displaystyle \frac{i}{\exp(2iz)+1}-\frac{i\exp(2iz)}{\exp(2iz)+1}$

\subsection*{exptanh($z$)}

Returns the hyperbolic tangent of $z$ in exponential form.

{\color{blue}
\begin{verbatim}
exptanh(z)
\end{verbatim}
}

\noindent
$\displaystyle -\frac{1}{\exp(2z)+1}+\frac{\exp(2z)}{\exp(2z)+1}$

\subsection*{factorial($n$)}

Returns the factorial of $n$.
The expression {\tt n!} can also be used.

{\color{blue}
\begin{verbatim}
20!
\end{verbatim}
}

\noindent
$2432902008176640000$

\subsection*{float($x$)}

Returns expression $x$ with rational numbers and integers converted to
floating point values.
The symbol {\tt pi} and the natural number are also converted.

{\color{blue}
\begin{verbatim}
float(212^17)
\end{verbatim}
}

\noindent
$\displaystyle 3.52947\times 10^{39}$

\subsection*{floor($x$)}

Returns the largest integer less than or equal to $x$.

{\color{blue}
\begin{verbatim}
floor(1/2)
\end{verbatim}
}

\noindent
$0$

\subsection*{for($i,j,k,a,b,\ldots$)}

For $i$ equals $j$ through $k$ evaluate $a$, $b$, etc.

{\color{blue}
\begin{verbatim}
for(k,1,3,A=k,print(A))
\end{verbatim}
}

\noindent
$A=1$\\
$A=2$\\
$A=3$

\bigskip
\noindent
Note: The original value of $i$ is restored after {\tt for} completes.
If symbol {\tt i} is used for index variable $i$
then the imaginary unit is overridden in the scope of {\tt for}.

\subsection*{i}

Symbol {\tt i} is initialized to the imaginary unit $\sqrt{-1}$.

{\color{blue}
\begin{verbatim}
exp(i pi)
\end{verbatim}
}

\noindent
$-1$

\bigskip
\noindent
Note: It is OK to clear or redefine {\tt i} and use the symbol for something else.

\subsection*{imag($z$)}

Returns the imaginary part of complex $z$.

{\color{blue}
\begin{verbatim}
imag(2 - 3i)
\end{verbatim}
}

\noindent
$-3$

\subsection*{inner($a,b,\ldots$)}

Returns the inner product of vectors, matrices, and tensors.
Also known as the matrix product.

{\color{blue}
\begin{verbatim}
A = ((a,b),(c,d))
B = (x,y)
inner(A,B)
\end{verbatim}
}

\noindent
$\displaystyle
\begin{bmatrix}
ax+by\\
cx+dy
\end{bmatrix}
$

\bigskip
\noindent
Note: {\tt inner} and {\tt dot} are the same function.

\subsection*{integral($f,x$)}

Returns the integral of $f$ with respect to $x$.

{\color{blue}
\begin{verbatim}
integral(x^2,x)
\end{verbatim}
}

\noindent
$\displaystyle \tfrac{1}{3}x^3$

\subsection*{inv($m$)}

Returns the inverse of matrix $m$.

{\color{blue}
\begin{verbatim}
A = ((1,2),(3,4))
inv(A)
\end{verbatim}
}

\noindent
$\displaystyle
\begin{bmatrix}
-2 & 1\\
\tfrac{3}{2} & -\tfrac{1}{2}
\end{bmatrix}
$

\subsection*{j}

Set {\tt j=sqrt(-1)} to use {\tt j} for the imaginary unit instead of {\tt i}.

{\color{blue}
\begin{verbatim}
j = sqrt(-1)
1/sqrt(-1)
\end{verbatim}
}

\noindent
$-j$

\subsection*{last}

The result of the previous calculation is stored in {\tt last}.

{\color{blue}
\begin{verbatim}
212^17
\end{verbatim}
}

\noindent
$3529471145760275132301897342055866171392$

{\color{blue}
\begin{verbatim}
last
\end{verbatim}
}

\noindent
$last=3529471145760275132301897342055866171392$

\bigskip
\noindent
Symbol {\tt last} is an implied argument when a function has no argument list.

{\color{blue}
\begin{verbatim}
float
\end{verbatim}
}

\noindent
$\displaystyle 3.52947\times10^{39}$

\subsection*{log($x$)}

Returns the natural logarithm of $x$.

{\color{blue}
\begin{verbatim}
log(x^y)
\end{verbatim}
}

\noindent
$y\log(x)$

\subsection*{mag($z$)}

Returns the magnitude of complex $z$.
Function {\tt mag} treats undefined symbols as real while {\tt abs} does not.

{\color{blue}
\begin{verbatim}
mag(x + i y)
\end{verbatim}
}

\noindent
$\displaystyle (x^2+y^2)^{1/2}$

\subsection*{minor($m,i,j$)}

Returns the minor of square matrix $m$ for row $i$ and column $j$.

{\color{blue}
\begin{verbatim}
A = ((1,2,3),(4,5,6),(7,8,9))
minor(A,1,1) == det(minormatrix(A,1,1))
\end{verbatim}
}

\noindent
1

\subsection*{minormatrix($m,i,j$)}

Returns a copy of matrix $m$ with row $i$ and column $j$ removed.

{\color{blue}
\begin{verbatim}
A = ((1,2,3),(4,5,6),(7,8,9))
minormatrix(A,1,1)
\end{verbatim}
}

\noindent
$
\begin{bmatrix}
5&6\cr
8&9
\end{bmatrix}
$

\subsection*{not($x$)}

Returns 0 if $x$ is true (nonzero).
Returns 1 otherwise.

{\color{blue}
\begin{verbatim}
not(1=1)
\end{verbatim}
}

\noindent
$0$

\subsection*{numerator($x$)}

Returns the numerator of expression $x$.

{\color{blue}
\begin{verbatim}
numerator(a/b)
\end{verbatim}
}

\noindent
$a$

\subsection*{or($a,b,\ldots$)}

Returns 1 if at least one argument is true (nonzero).
Returns 0 otherwise.

{\color{blue}
\begin{verbatim}
or(1=1,2=2)
\end{verbatim}
}

\noindent
$1$

\subsection*{outer($a,b,\ldots$)}

Returns the outer product of vectors, matrices, and tensors.

{\color{blue}
\begin{verbatim}
A = (a,b,c)
B = (x,y,z)
outer(A,B)
\end{verbatim}
}

\noindent
$\displaystyle
\begin{bmatrix}
a x & a y & a z\\
b x & b y & b z\\
c x & c y & c z
\end{bmatrix}
$

\subsection*{pi}

Symbol for $\pi$.

{\color{blue}
\begin{verbatim}
exp(i pi)
\end{verbatim}
}

\noindent
$-1$

\subsection*{polar($z$)}

Returns complex $z$ in polar form.

{\color{blue}
\begin{verbatim}
polar(x - i y)
\end{verbatim}
}

\noindent
$\displaystyle (x^2+y^2)^{1/2}\exp(i\arctan(-y,x))$

\subsection*{power}

Use \verb$^$ to raise something to a power.
Use parentheses for negative powers.

{\color{blue}
\begin{verbatim}
x^(-2)
\end{verbatim}
}

\noindent
$\displaystyle \frac{1}{x^2}$

\subsection*{print($a,b,\ldots$)}

Evaluate expressions and print the results.
Useful for printing from inside a {\tt for} loop.

{\color{blue}
\begin{verbatim}
for(j,1,3,print(j))
\end{verbatim}
}

\noindent
$j=1$\newline
$j=2$\newline
$j=3$

\section*{product($i,j,k,f$)}

For $i$ equals $j$ through $k$ evaluate $f$.
Returns the product of all $f$.

{\color{blue}
\begin{verbatim}
product(j,1,3,x + j)
\end{verbatim}
}

\noindent
$\displaystyle x^3+6x^2+11x+6$

\bigskip
\noindent
Note: The original value of $i$ is restored after {\tt product} completes.
If symbol {\tt i} is used for index variable $i$
then the imaginary unit is overridden in the scope of {\tt product}.

\subsection*{quote($x$)}

Returns expression $x$ without evaluating it first.

{\color{blue}
\begin{verbatim}
quote((x + 1)^2)
\end{verbatim}
}

\noindent
$\displaystyle (x+1)^2$

\subsection*{rank($a$)}

Returns the number of indices that tensor $a$ has.

{\color{blue}
\begin{verbatim}
A = ((a,b),(c,d))
rank(A)
\end{verbatim}
}

\noindent
2

\subsection*{rationalize($x$)}

Returns expression $x$ with everything over a common denominator.

{\color{blue}
\begin{verbatim}
rationalize(1/a + 1/b + 1/2)
\end{verbatim}
}

\noindent
$\displaystyle \frac{2a+ab+2b}{2ab}$

\bigskip
\noindent
Note:
\verb$rationalize$
returns an unexpanded expression.
If the result is assigned to a symbol, evaluating the symbol will expand the result.
Use
\verb$binding$
to retrieve the unexpanded expression.

{\color{blue}
\begin{verbatim}
f = rationalize(1/a + 1/b + 1/2)
binding(f)
\end{verbatim}
}

\noindent
$\displaystyle \frac{2a+ab+2b}{2ab}$

\subsection*{real($z$)}

Returns the real part of complex $z$.

{\color{blue}
\begin{verbatim}
real(2 - 3i)
\end{verbatim}
}

\noindent
2

\subsection*{rect($z$)}

Returns complex $z$ in rectangular form.

{\color{blue}
\begin{verbatim}
rect(exp(i x))
\end{verbatim}
}

\noindent
$\displaystyle \cos(x)+i\sin(x)$

\subsection*{run({\it file})}

Run script {\it file}.
Useful for importing function libraries.

{\color{blue}
\begin{verbatim}
run("Downloads/EVA.txt")
\end{verbatim}
}

\noindent
Note: {\it file} must be in the Downloads folder due to security requirements for apps distributed on the Mac App Store.

\subsection*{simplify($x$)}

Returns expression $x$ in a simpler form.

{\color{blue}
\begin{verbatim}
simplify(sin(x)^2 + cos(x)^2)
\end{verbatim}
}

\noindent
1

\subsection*{sin($x$)}

Returns the sine of $x$.

{\color{blue}
\begin{verbatim}
sin(pi/4)
\end{verbatim}
}

\noindent
$\displaystyle \frac{1}{2^{1/2}}$

\subsection*{sinh($x$)}

Returns the hyperbolic sine of $x$.

{\color{blue}
\begin{verbatim}
circexp(sinh(x))
\end{verbatim}
}

\noindent
$\displaystyle -\tfrac{1}{2}\exp(-x)+\tfrac{1}{2}\exp(x)$

\subsection*{sqrt($x$)}

Returns the square root of $x$.

{\color{blue}
\begin{verbatim}
sqrt(10!)
\end{verbatim}
}

\noindent
$\displaystyle 720\; 7^{1/2}$

\subsection*{stop}

In a script, it does what it says.

\subsection*{sum($i,j,k,f$)}

For $i$ equals $j$ through $k$ evaluate $f$.
Returns the sum of all $f$.

{\color{blue}
\begin{verbatim}
sum(j,1,5,x^j)
\end{verbatim}
}

\noindent
$\displaystyle x^5+x^4+x^3+x^2+x$

\bigskip
\noindent
Note: The original value of $i$ is restored after {\tt sum} completes.
If symbol {\tt i} is used for index variable $i$
then the imaginary unit is overridden in the scope of {\tt sum}.

\subsection*{tan($x$)}

Returns the tangent of $x$.

{\color{blue}
\begin{verbatim}
simplify(tan(x) - sin(x)/cos(x))
\end{verbatim}
}

\noindent
0

\subsection*{tanh($x$)}

Returns the hyperbolic tangent of $x$.

{\color{blue}
\begin{verbatim}
circexp(tanh(x))
\end{verbatim}
}

\noindent
$\displaystyle -\frac{1}{\exp(2x)+1}+\frac{\exp(2x)}{\exp(2x)+1}$

\subsection*{test($a,b,c,d,\ldots$)}

If argument $a$ is true (nonzero) then $b$ is returned, else if $c$ is true then $d$ is returned, etc.
If the number of arguments is odd then the final argument is returned if all else fails.
Expressions can include the relational operators
\verb$=$,
\verb$==$,
\verb$<$,
\verb$<=$,
\verb$>$,
\verb$>=$.
Use the
\verb$not$
function to test for inequality.
(The equality operator
\verb$==$
is available for contexts in which
\verb$=$
is the assignment operator.)

{\color{blue}
\begin{verbatim}
A = 1
B = 1
test(A=B,"yes","no")
\end{verbatim}
}

\noindent
yes

\subsection*{trace}

Set {\tt trace=1} in a script to print the script as it is evaluated.
Useful for debugging.

{\color{blue}
\begin{verbatim}
trace = 1
\end{verbatim}
}

\noindent
Note:
The
\verb$contract$
function is used to obtain the trace of a matrix.

\subsection*{transpose($a,i,j$)}

Returns the transpose of tensor $a$ with respect to indices $i$ and $j$.
If $i$ and $j$ are omitted then 1 and 2 are used.
Hence a matrix can be transposed with a single argument.

{\color{blue}
\begin{verbatim}
A = ((a,b),(c,d))
transpose(A)
\end{verbatim}
}

\noindent
$\displaystyle
\begin{bmatrix}
a & c\\
b & d
\end{bmatrix}
$

\bigskip
\noindent
Note:
The argument list can be extended for multiple transpose operations.
The arguments are evaluated from left to right.
For example,
\verb$transpose(A,1,2,2,3)$
is equivalent to
\verb$transpose(transpose(A,1,2),2,3)$.

\subsection*{unit($n$)}

Returns an $n$ by $n$ identity matrix.

{\color{blue}
\begin{verbatim}
unit(3)
\end{verbatim}
}

\noindent
$\displaystyle
\begin{bmatrix}
1 & 0 & 0\\
0 & 1 & 0\\
0 & 0 & 1
\end{bmatrix}
$

\subsection*{zero($i,j,\ldots$)}

Returns a null tensor with dimensions $i$, $j$, etc.
Useful for creating a tensor and then setting component values.

{\color{blue}
\begin{verbatim}
A = zero(3,3)
for(k,1,3,A[k,k]=k)
A
\end{verbatim}
}

\noindent
$\displaystyle
A=
\begin{bmatrix}
1 & 0 & 0\\
0 & 2 & 0\\
0 & 0 & 3
\end{bmatrix}
$
