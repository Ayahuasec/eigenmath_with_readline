\subsection{Arithmetic}

\noindent
Integers and rational numbers can have any number of digits,
regardless of native word size.
For example, $212^{17}\gg64\,\text{bits}$.

\begin{Verbatim}[formatcom=\color{blue}]
2^64
\end{Verbatim}

\noindent
$\displaystyle 18446744073709551616$

\begin{Verbatim}[formatcom=\color{blue}]
212^17
\end{Verbatim}

\noindent
$\displaystyle 3529471145760275132301897342055866171392$

\bigskip
\noindent
Integer and rational number arithmetic is used by default.

\begin{Verbatim}[formatcom=\color{blue}]
1/2 + 1/3
\end{Verbatim}

\noindent
$\displaystyle \tfrac{5}{6}$

\bigskip
\noindent
Floating point arithmetic can also be used.

\begin{Verbatim}[formatcom=\color{blue}]
1/2 + 1/3.0
\end{Verbatim}

\noindent
$\displaystyle 0.833333$

\bigskip
\noindent
An integer or rational number result can be converted to a floating
point value by entering {\it float}.

\begin{Verbatim}[formatcom=\color{blue}]
212^17
\end{Verbatim}

\noindent
$\displaystyle 3529471145760275132301897342055866171392$

\begin{Verbatim}[formatcom=\color{blue}]
float
\end{Verbatim}

\noindent
$\displaystyle 3.52947\times10^{39}$

\bigskip
\noindent
The following example shows how to enter a floating point value
using scientific notation.

\begin{Verbatim}[formatcom=\color{blue}]
epsilon = 1.0 10^(-6)
epsilon
\end{Verbatim}

\noindent
$\displaystyle \varepsilon=1.0\times10^{-6}$
