\subsection{Syntax}
Arithmetic operators have the expected precedence of
multiplication and division before addition and subtraction.
Subexpressions enclosed in parentheses have highest precedence.

\begin{center}
\begin{tabular}{clll}
{\it Math} & & {\it Eigenmath} & {\it Comment}
\\[1ex]
$a=b$ & & \verb$a == b$ & {\it test for equality}
\\[1ex]
$-a$ & & {\tt -a} & {\it negation}
\\[1ex]
$a+b$ & & {\tt a+b} & {\it addition}
\\[1ex]
$a-b$ & & {\tt a-b} & {\it subtraction}
\\[1ex]
$ab$ & & {\tt a b} & {\it multiplication, also} \verb$a*b$
\\[1ex]
$\displaystyle\frac{a}{b}$ & & {\tt a/b} & {\it division}
\\
\\
$\displaystyle\frac{a}{bc}$ & & {\tt a/b/c} & {\it division is left-associative}
\\
\\
$a^2$ & & {\tt a{\char94}2} & {\it power}
\\[1ex]
$\sqrt{a}$ & & \verb$sqrt(a)$ & {\it square root, also} \verb$a^(1/2)$
\\[1ex]
$a\,(b+c)$ & & {\tt a (b+c)} & {\it space is required}
\\[1ex]
$f(a)$ & & {\tt f(a)} & {\it function}
\\
\\
$\begin{pmatrix}a\\ b\\ c\end{pmatrix}$ & & {\tt (a,b,c)} & {\it vector}
\\
\\
$\begin{pmatrix}a&b\\ c&d\end{pmatrix}$ & & {\tt ((a,b),(c,d))} & {\it matrix}
\\\\
$F^1{}_2$ & & {\tt F[1,2]} & {\it tensor component access}
\\[1ex]
 & & \verb$"hello, world"$ & {\it string literal}
\\[1ex]
$\pi$ & & {\tt pi} &
\\[1ex]
$e$ && {\tt exp(1)} & {\it natural number}
\end{tabular}
\end{center}
