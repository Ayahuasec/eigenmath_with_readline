\subsection{Syntax}
The following table summarizes the various operators and expression syntax.
The arithmetic operators have the expected precedence of
multiplication and division before addition and subtraction.
Subexpressions surrounded by parentheses have highest precedence.

\begin{center}
\begin{tabular}{clll}
{\it Math} & & {\it Eigenmath} & {\it Comment} \\
\\
$a=b$ & & \verb$a == b$ & {\it test for equality} \\
\\
$-a$ & & {\tt -a} & {\it negation} \\
\\
$a+b$ & & {\tt a+b} & {\it addition} \\
\\
$a-b$ & & {\tt a-b} & {\it subtraction} \\
\\
$ab$ & & {\tt a b} & {\it multiplication, alternatively,} \verb$a*b$ \\
\\
$\displaystyle\frac{a}{b}$ & & {\tt a/b} & {\it division}\\
\\
$\displaystyle\frac{a}{bc}$ & & {\tt a/b/c} & {\it division operator is left-associative} \\
\\
$a^2$ & & {\tt a{\char94}2} & {\it power}\\
\\
$\sqrt{a}$ & & \verb$sqrt(a)$ & {\it square root, alternatively,} \verb$a^(1/2)$ \\
\\
$a(b+c)$ & & {\tt a (b+c)} & {\it note the space in between, alternatively,} \verb$a*(b+c)$ \\
\\
$f(a)$ & & {\tt f(a)} & {\it function} \\
\\
$\begin{pmatrix}a\\ b\\ c\end{pmatrix}$ & & {\tt (a,b,c)} & {\it vector} \\
\\
$\begin{pmatrix}a&b\\ c&d\end{pmatrix}$ & & {\tt ((a,b),(c,d))} & {\it matrix} \\
\\
$F^1{}_2$ & & {\tt F[1,2]} & {\it tensor component access} \\
\\
 & & \verb$"hello, world"$ & {\it string literal} \\
\\
$\pi$ & & {\tt pi} & \\
\\
$e$ && {\tt exp(1)} & {\it natural number}
\end{tabular}
\end{center}
