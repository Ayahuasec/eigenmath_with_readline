
\bigskip
\noindent
Here is a useful trick.
Difficult integrals involving sine and cosine
can often be solved by using exponentials.
Trigonometric simplifications involving powers
and multiple angles turn into simple algebra in the
exponential domain.
For example, the definite integral
$$\int_0^{2\pi}\left(\sin^4t-2\cos^3(t/2)\sin t\right)dt$$
can be solved as follows.

\begin{Verbatim}[formatcom=\color{blue},samepage=true]
f = sin(t)^4-2*cos(t/2)^3*sin(t)
f = circexp(f)
defint(f,t,0,2*pi)
\end{Verbatim}

\noindent
$\displaystyle -\tfrac{16}{5}+\tfrac{3}{4}\pi$

\bigskip
\noindent
Here is a check of the result.

\begin{Verbatim}[formatcom=\color{blue},samepage=true]
g = integral(f,t)
f-d(g,t)
\end{Verbatim}

\noindent
$\displaystyle 0$
