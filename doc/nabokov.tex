\section{Introduction}

\noindent
Consider the following arithmetic from Vladimir Nabokov's autobiography ``Speak, Memory.''

\begin{quote}
A foolish tutor had explained logarithms to me much too early, and I had
read (in a British publication, the {\it Boy's Own Paper}, I believe)
about a certain Hindu calculator who in exactly two seconds could find the
seventeenth root of, say,
352947114576027513 2301897342055866171392
(I am not sure I have got this right; anyway the root was 212).
\end{quote}

\noindent
Let us compute $212^{17}$ and check the result.
At the Eigenmath prompt, enter

{\color{blue}
\begin{verbatim}
212^17
\end{verbatim}
}

\noindent
The result is

\bigskip
\noindent
$3529471145760275132301897342055866171392$

\bigskip
\noindent
Now let us compute the seventeenth root of this number.
{\color{blue}
\begin{verbatim}
N = 212^17
N
\end{verbatim}
}
$N=3529471145760275132301897342055866171392$
{\color{blue}
\begin{verbatim}
N^(1/17)
\end{verbatim}
}
212
