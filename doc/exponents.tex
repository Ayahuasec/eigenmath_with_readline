
\subsection{Exponents}
Eigenmath requires parentheses around negative exponents.
For example,

\begin{Verbatim}[formatcom=\color{blue}]
10^(-3)
\end{Verbatim}

instead of

\begin{Verbatim}[formatcom=\color{blue}]
10^-3
\end{Verbatim}

The reason for this is that the binding of the negative sign is not always
obvious.
For example, consider

\begin{Verbatim}[formatcom=\color{blue}]
x^-1/2
\end{Verbatim}

It is not clear whether the exponent should be $-1$ or $-1/2$.
So Eigenmath requires

\begin{Verbatim}[formatcom=\color{blue}]
x^(-1/2)
\end{Verbatim}

which is unambiguous.
In general, parentheses are always required when the exponent
is an expresssion.
For example, \verb$x^1/2$ is evaluated as $(x^1)/2$ which
is probably not the desired result.

\begin{Verbatim}[formatcom=\color{blue}]
x^1/2
\end{Verbatim}

$\displaystyle \frac{1}{2}x$

Using \verb$x^(1/2)$ yields the desired result.

\begin{Verbatim}[formatcom=\color{blue}]
x^(1/2)
\end{Verbatim}

$\displaystyle x^{1/2}$
