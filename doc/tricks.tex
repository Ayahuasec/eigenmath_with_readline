\section{Tricks}
\begin{enumerate}

\item
In a script, line breaking is allowed provided the line breaks occur immediately after operators.
The scanner will automatically go to the next line after an operator.

\item
Setting \verb$trace=1$ in a script causes each line to be printed just before it is evaluated.
This is useful for debugging.

\item
The last result is stored in the symbol $last$.

\item
Use \verb$contract(A)$ to get the mathematical trace of matrix $A$.

\item
Use \verb$binding(s)$ to get the unevaluated binding of symbol $s$.

\item
Use \verb$s=quote(s)$ to clear symbol $s$.

\item
Use \verb$float(pi)$ to get the floating point value of $\pi$.
Set \verb$pi=float(pi)$ to evaluate expressions with a numerical value for $\pi$.
Set \verb$pi=quote(pi)$ to make $\pi$ symbolic again.

\item
Assign strings to unit names so they are printed normally.
For example, setting \verb$meter="meter"$ causes the symbol {\it meter}
to be printed as meter instead of $m_{eter}$.

\item
Use \verb$expsin$ and \verb$expcos$ instead of \verb$sin$ and \verb$cos$.
Trigonometric simplifications occur automatically when exponentials are used.

\item
Use \verb$A==B$ or \verb$A-B==0$ to test for equality of $A$ and $B$.
The equality operator \verb$==$ uses a cross multiply algorithm to eliminate denominators.
Hence \verb$==$ can typically determine equality even when the unsimplified result of $A-B$ is nonzero.
Note: Equality tests involving floating point numbers can be problematic
due to roundoff error.

\item
If local symbols are needed in a function, they can be appended to {\it arg-list}.
(The caller does not have to supply all the arguments.)
The following example uses Rodrigues's formula to
compute an associated Legendre function of $\cos\theta$.
\begin{equation*}
P_n^m(x)=\frac{1}{2^n\,n!}(1-x^2)^{m/2}\frac{d^{n+m}}{dx^{n+m}}(x^2-1)^n
\end{equation*}
Function $P$ below first computes $P_n^m(x)$ for local variable
$x$ and then uses {\it eval} to replace $x$ with $f$.
In this case, $f=\cos\theta$.

\begin{Verbatim}[formatcom=\color{blue}]
x = 123 -- global x in use, need local x in P
P(f,n,m,x) = eval(1/(2^n n!) (1 - x^2)^(m/2) d((x^2 - 1)^n,x,n + m),x,f)
P(cos(theta),2,0) -- arguments f, n, m, but not x
\end{Verbatim}

\noindent
$\displaystyle \tfrac{3}{2} \cos(\theta)^2-\tfrac{1}{2}$

\bigskip
\noindent
Note: The maximum number of arguments is nine.

\end{enumerate}
