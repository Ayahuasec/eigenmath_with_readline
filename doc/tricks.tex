
\section{Tricks}
\begin{enumerate}

\item
The Eigenmath result field can be copied to the pasteboard by
click, drag, then release.

\item
The last result is stored in the symbol {\tt last}.

\item
In a script, setting {\tt trace=1}
causes each line to be printed just before it is evaluated.
Useful for debugging.

\item
Use {\tt contract(A)} to get the mathematical trace of matrix $A$.

\item
Calculations in a script can span multiple lines.
The trick is to arrange things so the parser will keep going.
For example, if a calculation ends with a plus sign, the parser will go to the next line to get another term.
Also, the parser will keep going when it expects a close parenthesis.

\item
Normally a function body is not evaluated when a function is defined.
However, in some cases it is required that the function body be the result of something.
The trick is to use eval.
For example, the following code causes the function body to be a sixth order Taylor series expansion of $cos(x)$.

\begin{Verbatim}[formatcom=\color{blue}]
f(x) = eval(taylor(cos(x),x,6))
\end{Verbatim}

\item
Use {\tt binding} to see the unevaluated binding of a symbol.

\begin{Verbatim}[formatcom=\color{blue}]
binding(f)
\end{Verbatim}

\item
This is how to clear a symbol.

\begin{Verbatim}[formatcom=\color{blue}]
f = quote(f)
\end{Verbatim}

\end{enumerate}
