
\section{Tricks}
\begin{enumerate}

\item
The last result is stored in the symbol {\tt last}.

\item
In a script, setting {\tt trace=1}
causes each line to be printed just before it is evaluated.
Useful for debugging.

\item
Use {\tt contract(A)} to get the mathematical trace of matrix \verb$A$.

\item
Calculations in a script can span multiple lines.
The trick is to arrange things so the parser will keep going.
For example, if a calculation ends with a plus sign, the parser will go to the next line to get another term.
Also, the parser will keep going when it expects a close parenthesis.

\item
Normally a function body is not evaluated when a function is defined.
However, in some cases it is required that the function body be the result of something.
The trick is to use \verb$eval$.
For example, use \verb$f(x)=eval(taylor(cos(x),x,6))$
to define \verb$f(x)$ as the sixth order Taylor series expansion of $\cos(x)$.

\item
Use \verb$binding(f)$ to get the unevaluated binding of symbol \verb$f$.

\item
Use \verb$f=quote(f)$ to clear symbol \verb$f$.

\item
Use \verb$float(pi)$ to get the floating point value of $\pi$.

\end{enumerate}
