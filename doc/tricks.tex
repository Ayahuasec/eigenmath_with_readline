\section{Tricks}
\begin{enumerate}

\item
In the result display, a click-drag-release will copy a selection of the display to the clipboard.

\item
In a script, line breaking is allowed provided the line breaks occur immediately after operators.
The scanner will automatically go to the next line after an operator.

\item
Setting \verb$trace=1$ in a script causes each line to be printed just before it is evaluated.
This is useful for debugging.

\item
The last result is stored in the symbol $last$.

\item
Use \verb$contract(A)$ to get the mathematical trace of matrix $A$.

\item
Use \verb$binding(s)$ to get the unevaluated binding of symbol $s$.

\item
Use \verb$s=quote(s)$ to clear symbol $s$.

\item
Use \verb$float(pi)$ to get the floating point value of $\pi$.

\item
Assign strings to unit names so they are printed normally.
For example, setting \verb$meter="meter"$ causes the symbol {\it meter}
to be printed as meter instead of $m_{eter}$.

\item
Use \verb$expsin$ and \verb$expcos$ instead of \verb$sin$ and \verb$cos$.
Trigonometric simplifications occur automatically when exponentials are used.

\item
Use \verb$A==B$ or \verb$A-B==0$ to test for equality of $A$ and $B$.
The equality operator \verb$==$ uses a cross multiply algorithm to eliminate denominators.
Hence \verb$==$ can typically determine equality even when the unsimplified result of $A-B$ is nonzero.

\end{enumerate}
