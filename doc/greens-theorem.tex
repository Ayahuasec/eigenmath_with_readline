
\subsection{Green's theorem}
Green's theorem tells us that
$$\oint P\,dx+Q\,dy=\int\!\!\!\int
\left(\frac{\partial Q}{\partial x}-\frac{\partial P}{\partial y}\right)
dx\,dy$$

\noindent
In other words, a line integral and a surface integral can yield
the same result.

\bigskip
\noindent
Example 1.
The following exercise is from {\it Advanced Calculus}
by Wilfred Kaplan, p.~287.
Evaluate $\oint (2x^3-y^3)\,dx+(x^3+y^3)\,dy$ around the circle
$x^2+y^2=1$ using Green's theorem.

\bigskip
\noindent
It turns out that Eigenmath cannot solve the double integral over
$x$ and $y$ directly.
Polar coordinates are used instead.

\begin{Verbatim}[formatcom=\color{blue},samepage=true]
P = 2x^3 - y^3
Q = x^3 + y^3
f = d(Q,x) - d(P,y)
x = r cos(theta)
y = r sin(theta)
defint(f r,r,0,1,theta,0,2pi)
\end{Verbatim}

\noindent
$\displaystyle \tfrac{3}{2}\pi$

\bigskip
\noindent
The $defint$ integrand is $f\,r$ because $r\,dr\,d\theta=dx\,dy$.

\bigskip
\noindent
Now let us try computing the line integral side of Green's theorem
and see if we get the same result.
We need to use the trick of converting sine and cosine to exponentials
so that Eigenmath can find a solution.

\begin{Verbatim}[formatcom=\color{blue},samepage=true]
x = cos(t)
y = sin(t)
P = 2x^3 - y^3
Q = x^3 + y^3
f = P d(x,t) + Q d(y,t)
f = circexp(f)
defint(f,t,0,2pi)
\end{Verbatim}

\noindent
$\displaystyle \tfrac{3}{2}\pi$

\bigskip
\noindent
Example 2.
Compute both sides of Green's theorem for
$F=(1-y,x)$ over the disk $x^2+y^2\le4$.

\bigskip
\noindent
First compute the line integral along the boundary of the disk.
Note that the radius of the disk is 2.

\begin{Verbatim}[formatcom=\color{blue},samepage=true]
-- Line integral
P = 1 - y
Q = x
x = 2 cos(t)
y = 2 sin(t)
defint(P d(x,t) + Q d(y,t),t,0,2pi)
\end{Verbatim}

\noindent
$\displaystyle 8\pi$

\begin{Verbatim}[formatcom=\color{blue},samepage=true]
-- Surface integral
x = quote(x) --clear x
y = quote(y) --clear y
h = sqrt(4-x^2)
defint(d(Q,x) - d(P,y),y,-h,h,x,-2,2)
\end{Verbatim}

\noindent
$\displaystyle 8\pi$

\begin{Verbatim}[formatcom=\color{blue},samepage=true]
-- Try computing the surface integral using polar coordinates.
f = d(Q,x) - d(P,y) -- do before change of coordinates
x = r cos(theta)
y = r sin(theta)
defint(f r,r,0,2,theta,0,2pi)
\end{Verbatim}

\noindent
$\displaystyle 8\pi$

\begin{Verbatim}[formatcom=\color{blue},samepage=true]
defint(f r,theta,0,2pi,r,0,2) -- try integrating over theta first
\end{Verbatim}

\noindent
$\displaystyle 8\pi$

\bigskip
\noindent
In this case, Eigenmath solved both forms of the polar integral.
However, in cases where Eigenmath fails to solve a double integral, try
changing the order of integration.
