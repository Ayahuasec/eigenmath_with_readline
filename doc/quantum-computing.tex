A quantum computer can be simulated by applying rotations to a
unit vector
$u\in\mathbb{C}^{2^n}$ where $\mathbb{C}$ is the set of complex numbers
and $n$ is the number of qubits.
The dimension is $2^n$ because a register with $n$ qubits
has $2^n$ eigenstates.
Quantum operations are ``rotations'' because they preserve $|u|=1$.
Mathematically, a rotation of $u$ is equivalent to the product $Ru$
where $R$ is a $2^n\times2^n$ matrix.

\bigskip
\noindent
Eigenstates $|j\rangle$ are represented by the following vectors.
(Each vector has $2^n$ elements.)
\begin{align*}
&|0\rangle=(1,0,0,\dots,0)
\\
&|1\rangle=(0,1,0,\ldots,0)
\\
&|2\rangle=(0,0,1,\ldots,0)
\\
&\vdots
\\
&|2^n-1\rangle=(0,0,0,\ldots,1)
\end{align*}

\noindent
A quantum computer algorithm is a sequence of rotations
applied to the initial state $|0\rangle$.
(The sequence could be combined into a single rotation
by associativity of matrix multiplication.)
Let $\psi_f$ be the final state of the quantum computer
after all the rotations have been applied.
Like any other state, $\psi_f$ is a linear combination of eigenstates.
\begin{equation*}
\psi_f=\sum_{j=0}^{2^n-1}c_j|j\rangle,\quad|\psi_f|=1
\end{equation*}

\noindent
The last step is to measure $\psi_f$ and get a result.
Measurement rotates $\psi_f$ to an eigenstate $|j\rangle$.
The measurement result is $|j\rangle$.
The probability $P_j$ of getting a specific result $|j\rangle$ is
\begin{equation*}
P_j=|c_j|^2=c_jc_j^*
\end{equation*}

\noindent
Note that if $\psi_f$ is already an eigenstate then no rotation occurs.
(The probability of rotating to a different eigenstate is zero.)
Since the measurement result is always an eigenstate,
the coefficients $c_j$ cannot be observed.
However, the same calculation can be run multiple times
to obtain a probability distribution of results.
The probability distribution is an estimate
of $|c_j|^2$ for each $|j\rangle$ in $\psi_f$.

\bigskip
\noindent
Unlike a real quantum computer, in a simulation
the final state $\psi_f$, or any other state, is available for inspection.
Hence there is no need to simulate the measurement process.
The probability distribution of the result can be computed directly as
\begin{equation*}
P=\psi_f\odot\psi_f^*
\end{equation*}

\noindent
where operator $\odot$ indicates a Hadamard (element-wise) product.
The result $P$ is a vector with component $P_j$ the
probability of eigenstate $|j\rangle$ and
\begin{equation*}
\sum P_j=1
\end{equation*}

\noindent
The Eigenmath function
$rotate(u,s,k,\ldots)$
rotates vector $u$ and returns the result.
Vector $u$ is required to have $2^n$ elements where $n$ is an
integer from 1 to 15.
Arguments $s,k,\ldots$ are a sequence of rotation codes
where $s$ is an upper case letter and $k$ is a qubit number
from 0 to $n-1$.
Rotations are evaluated from left to right.
The available rotation codes are

\begin{center}
\begin{tabular}{ll}
$C,k$ & Control prefix
\\
$H,k$ & Hadamard
\\
$P,k,\phi$ & Phase modifier (use $\phi=\tfrac{1}{4}\pi$ for $T$ rotation)
\\
$Q,k$ & Quantum Fourier transform
\\
$V,k$ & Inverse quantum Fourier transform
\\
$W,k,j$ & Swap bits
\\
$X,k$ & Pauli X
\\
$Y,k$ & Pauli Y
\\
$Z,k$ & Pauli Z
\end{tabular}
\end{center}

\noindent
Control prefix $C,k$ modifies the next rotation code so that it
is a controlled rotation with $k$ as the control qubit.
Use two or more prefixes to specify multiple control qubits.
For example, $C,k,C,j,X,m$ is a Toffoli rotation.
Fourier rotations $Q,k$ and $V,k$ are applied to qubits 0 through $k$.
($Q$ and $V$ ignore any control prefix.)

\bigskip
\noindent
Error codes
\begin{itemize}
\item[1] Argument $u$ is not a vector or does not have $2^n$ elements where $n=1,2,\ldots,15$.
\item[2] Unexpected end of argument list (i.e., missing argument).
\item[3] Bit number format error or range error.
\item[4] Unknown rotation code.
\end{itemize}

\bigskip
\noindent
Here are some useful Eigenmath code snippets for setting up a simulation
and computing the result.

\bigskip
\noindent
1. Initialize $\psi=|0\rangle$.
{\color{blue}
\begin{verbatim}
n = 4           -- number of qubits (example)
N = 2^n         -- number of eigenstates
psi = zero(N)
psi[1] = 1
\end{verbatim}}

\noindent
2. Compute the probability distribution for state $\psi$.
{\color{blue}
\begin{verbatim}
P = psi conj(psi)
\end{verbatim}}

\noindent
Hence
\begin{align*}
&\text{\tt P[1]}=\text{probability that $|0\rangle$ will be the result}
\\
&\text{\tt P[2]}=\text{probability that $|1\rangle$ will be the result}
\\
&\text{\tt P[3]}=\text{probability that $|2\rangle$ will be the result}
\\
&\vdots
\\
&\text{\tt P[N]}=\text{probability that $|N-1\rangle$ will be the result}
\end{align*}

\noindent
3. Draw a probability distribution.
{\color{blue}
\begin{verbatim}
xrange = (0,N)
yrange = (0,1)
draw(P[ceiling(x)],x)
\end{verbatim}}

\noindent
4. Compute an expectation value.
{\color{blue}
\begin{verbatim}
sum(k,1,N, (k - 1) P[k])
\end{verbatim}}

\noindent
5. Make the high order qubit ``don't care.''
{\color{blue}
\begin{verbatim}
for(k,1,N/2, P[k] = P[k] + P[k + N/2])
\end{verbatim}}

\noindent
Hence for $N=16$
\begin{align*}
&\text{\tt P[1]}=\text{probability that the result will be $|0\rangle$ or $|8\rangle$}
\\
&\text{\tt P[2]}=\text{probability that the result will be $|1\rangle$ or $|9\rangle$}
\\
&\text{\tt P[3]}=\text{probability that the result will be $|2\rangle$ or $|10\rangle$}
\\
&\vdots
\\
&\text{\tt P[8]}=\text{probability that the result will be $|7\rangle$ or $|15\rangle$}
\end{align*}

\noindent
Example.
{\color{blue}
\begin{verbatim}
-- Verify the following truth table for cnot

-- Input Output
--    00     00
--    01     11
--    10     10
--    11     01

U(psi) = rotate(psi,C,0,X,1) -- cnot

ket00 = (1,0,0,0)
ket01 = (0,1,0,0)
ket10 = (0,0,1,0)
ket11 = (0,0,0,1)

U(ket00) == ket00
U(ket01) == ket11
U(ket10) == ket10
U(ket11) == ket01
\end{verbatim}}
