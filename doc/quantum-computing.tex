A quantum computer can be simulated by applying rotations to a
unit vector
$u\in\mathbb{C}^{2^n}$ where $n$ is the number of qubits.
For example, four qubits would have $u\in\mathbb{C}^{16}$.
The dimension is $2^n$ because a register with $n$ qubits
has $2^n$ eigenstates.
Quantum operations are ``rotations'' because they preserve $|u|=1$.

\bigskip
\noindent
The Eigenmath function
$rotate(u,s,k,\ldots)$ rotates vector $u$ and returns the result.
Vector $u$ is required to have $2^n$ elements where $n$ is an
integer from 1 to 15.
Arguments $s,k,\ldots$ are a sequence of rotation codes
where $s$ is an upper case letter and $k$ is a qubit number
from 0 to $n-1$.
Rotations are evaluated from left to right.
The available rotation codes are

\begin{center}
\begin{tabular}{ll}
$C,k$ & Control prefix
\\
$H,k$ & Hadamard
\\
$P,k,\phi$ & Phase modifier
\\
$Q,k$ & Quantum Fourier transform
\\
$V,k$ & Inverse quantum Fourier transform
\\
$W,k,j$ & Swap bits
\\
$X,k$ & Pauli X
\\
$Y,k$ & Pauli Y
\\
$Z,k$ & Pauli Z
\end{tabular}
\end{center}

\noindent
Control prefix $C,k$ modifies the next rotation code so that it
is a controlled rotation with $k$ is the controlling qubit.
Fourier rotations $Q,k$ and $V,k$ are applied to qubits 0 through $k$.
($Q$ and $V$ ignore any control prefix.)

\bigskip
\noindent
Error codes
\begin{itemize}
\item[1] Argument $u$ is not a vector or does not have $2^n$ elements where $n=1,2,\ldots,15$.
\item[2] Unexpected end of argument list (i.e., missing argument).
\item[3] Bit number format error or range error.
\item[4] Unknown rotation code.
\end{itemize}

\bigskip
\noindent
Eigenstates $|j\rangle$ are represented by the following vectors.
(Each vector has $2^n$ elements.)
\begin{align*}
&|0\rangle=(1,0,0,\dots,0)
\\
&|1\rangle=(0,1,0,\ldots,0)
\\
&|2\rangle=(0,0,1,\ldots,0)
\\
&\vdots
\\
&|2^n-1\rangle=(0,0,0,\ldots,1)
\end{align*}

\noindent
A quantum computing algorithm is a sequence of rotations
applied to the initial state $|0\rangle$.
(Mathematically, the sequence can be collapsed into a single rotation.)
Let $\psi$ be the final state of the quantum computer
after all the rotations have been applied.
Like any other state, $\psi$ is a linear combination of eigenstates.
\begin{equation*}
\psi=\sum_{j=0}^{2^n-1}c_j|j\rangle,\quad|\psi|=1
\end{equation*}
The final step is to measure $\psi$ which will
change it to an eigenstate $|j\rangle$ with
probability
\begin{equation*}
P_j=c_jc_j^*
\end{equation*}
The output from a real quantum computer is always an eigenstate.
For example, the result of a two qubit quantum computer
is either $|0\rangle$, $|1\rangle$, $|2\rangle$, or $|3\rangle$.
In general the $c_j$'s cannot be observed directly.
However, we can run the same calculation multiple times
to obtain a probability distribution.
From the probability distribution we can estimate
$c_jc_j^*$ for each eigenstate.

\bigskip
\noindent
Eigenmath code snippets for before and after {\it rotate}:

\bigskip
\noindent
Initialize $\psi=|0\rangle$.
{\color{blue}
\begin{verbatim}
n = 4           -- number of qubits
N = 2^n         -- number of eigenstates
psi = zero(N)
psi[1] = 1
\end{verbatim}}

\noindent
Compute the probability distribution for state $\psi$.
{\color{blue}
\begin{verbatim}
P = psi conj(psi)
\end{verbatim}}

\noindent
Hence
\begin{align*}
&\text{\tt P[1]}=\text{probability that $|0\rangle$ will be the result}
\\
&\text{\tt P[2]}=\text{probability that $|1\rangle$ will be the result}
\\
&\text{\tt P[3]}=\text{probability that $|2\rangle$ will be the result}
\\
&\vdots
\\
&\text{\tt P[N]}=\text{probability that $|N-1\rangle$ will be the result}
\end{align*}

\noindent
Draw the probability distribution.
{\color{blue}
\begin{verbatim}
xrange = (0,N)
yrange = (0,1)
draw(P[ceiling(x)],x)
\end{verbatim}}

\noindent
Compute an expectation value.
{\color{blue}
\begin{verbatim}
sum(k,1,N, (k - 1) P[k])
\end{verbatim}}

\noindent
See ``Quantum Computing'' at eigenmath.org for examples.
