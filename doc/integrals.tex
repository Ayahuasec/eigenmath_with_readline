
\subsection{Integral}
$integral(f,x)$ returns the integral of $f$ with respect to $x$.
The $x$ can be omitted for expressions in $x$.
The argument list can be extended for multiple integrals.

\begin{Verbatim}[formatcom=\color{blue},samepage=true]
integral(x^2)
\end{Verbatim}

$\displaystyle \frac{1}{3}x^3$

\begin{Verbatim}[formatcom=\color{blue},samepage=true]
integral(x*y,x,y)
\end{Verbatim}

$\displaystyle \frac{1}{4}x^2y^2$

$defint(f,x,a,b,\ldots)$
computes the definite integral of $f$ with respect to $x$ evaluated from
$a$ to $b$.
The argument list can be extended for multiple integrals.
The following example computes the integral of $f=x^2$
over the domain of a semicircle.
For each $x$ along the abscissa, $y$ ranges from 0 to $\sqrt{1-x^2}$.

\begin{Verbatim}[formatcom=\color{blue},samepage=true]
defint(x^2,y,0,sqrt(1-x^2),x,-1,1)
\end{Verbatim}

$\displaystyle \frac{1}{8}\pi$

As an alternative, the $eval$ function can be used to compute a definite integral step by step.

\begin{Verbatim}[formatcom=\color{blue},samepage=true]
I = integral(x^2,y)
I = eval(I,y,sqrt(1-x^2))-eval(I,y,0)
I = integral(I,x)
eval(I,x,1)-eval(I,x,-1)
\end{Verbatim}

$\displaystyle \frac{1}{8}\pi$
