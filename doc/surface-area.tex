
\subsection{Surface area}
Let $S$ be a surface parameterized by $x$ and $y$.
That is, let $S=(x,y,z)$ where $z=f(x,y)$.
The tangent lines at a point on $S$ form a tiny parallelogram.
The area $a$ of the parallelogram is given by the magnitude of the cross product.
$$a=\left|\frac{\partial S}{\partial x}\times\frac{\partial S}{\partial y}\right|$$
By summing over all the parallelograms we obtain the total surface area $A$.
Hence
$$A=\int\!\!\!\int dA=\int\!\!\!\int a\,dx\,dy$$
The following example computes the surface area of a unit disk
parallel to the $xy$ plane.

\begin{Verbatim}[formatcom=\color{blue},samepage=true]
M1 = ((0,0,0),(0,0,-1),(0,1,0))
M2 = ((0,0,1),(0,0,0),(-1,0,0))
M3 = ((0,-1,0),(1,0,0),(0,0,0))
M = (M1,M2,M3)
cross(u,v) = dot(u,M,v)
z = 2
S = (x,y,z)
a = abs(cross(d(S,x),d(S,y)))
defint(a,y,-sqrt(1 - x^2),sqrt(1 - x^2),x,-1,1)
\end{Verbatim}

\noindent
$\displaystyle \pi$

\bigskip
\noindent
The result is $\pi$, the area of a unit circle, which is what we expect.
The following example computes the surface area of $z=x^2+2y$ over
a unit square.

\begin{Verbatim}[formatcom=\color{blue},samepage=true]
M1 = ((0,0,0),(0,0,-1),(0,1,0))
M2 = ((0,0,1),(0,0,0),(-1,0,0))
M3 = ((0,-1,0),(1,0,0),(0,0,0))
M = (M1,M2,M3)
cross(u,v) = dot(u,M,v)
z = x^2 + 2y
S = (x,y,z)
a = abs(cross(d(S,x),d(S,y)))
defint(a,x,0,1,y,0,1)
\end{Verbatim}

\noindent
$\displaystyle \tfrac{5}{8}\log(5)+\tfrac{3}{2}$

\bigskip
\noindent
The following exercise is from
{\it Multivariable Mathematics} by Williamson and Trotter, p. 598.
Find the area of the spiral ramp defined by
$$S=\begin{bmatrix}u\cos v\\\ u\sin v\\ v\end{bmatrix},\qquad 0\le u\le1,\qquad 0\le v\le3\pi$$

\begin{Verbatim}[formatcom=\color{blue},samepage=true]
M1 = ((0,0,0),(0,0,-1),(0,1,0))
M2 = ((0,0,1),(0,0,0),(-1,0,0))
M3 = ((0,-1,0),(1,0,0),(0,0,0))
M = (M1,M2,M3)
cross(u,v) = dot(u,M,v)
x = u cos(v)
y = u sin(v)
z = v
S = (x,y,z)
a = circexp(abs(cross(d(S,u),d(S,v))))
defint(a,u,0,1,v,0,3pi)
\end{Verbatim}

\noindent
$\displaystyle \tfrac{3}{2}\pi\log(1+2^{1/2})+\frac{3\pi}{2^{1/2}}$

\begin{Verbatim}[formatcom=\color{blue},samepage=true]
float
\end{Verbatim}

\noindent
$\displaystyle 10.8177$
