
\subsection{Curl in tensor form}
The curl of a vector function can be expressed in tensor form as
$$\mathop{\rm curl}{\bf F}=\epsilon_{ijk}\,{\partial F_k\over\partial x_j}$$
where $\epsilon_{ijk}$ is the Levi-Civita tensor.
The following script demonstrates that this formula is equivalent
to computing curl the old fashioned way.

\begin{Verbatim}[formatcom=\color{blue},samepage=true]
-- Define epsilon
epsilon = zero(3,3,3)
epsilon[1,2,3] = 1
epsilon[2,3,1] = 1
epsilon[3,1,2] = 1
epsilon[3,2,1] = -1
epsilon[1,3,2] = -1
epsilon[2,1,3] = -1
-- F is a generic vector function
F = (FX(),FY(),FZ())
-- A is the curl of F
A = outer(epsilon,d(F,(x,y,z)))
A = contract(A,3,4) --sum across k
A = contract(A,2,3) --sum across j
-- B is the curl of F computed the old fashioned way
BX = d(F[3],y)-d(F[2],z)
BY = d(F[1],z)-d(F[3],x)
BZ = d(F[2],x)-d(F[1],y)
B = (BX,BY,BZ)
-- Are A and B equal? Subtract to find out.
A-B
\end{Verbatim}

Here is the result when the script runs.

$\displaystyle \begin{bmatrix}0\\0\\0\end{bmatrix}$

The following is a variation on the previous script.
The product $\epsilon_{ijk}\,\partial F_k/\partial x_j$
is computed in just one line of code.
In addition, the outer product and the contraction across $k$
are now computed with a dot product.

\begin{Verbatim}[formatcom=\color{blue},samepage=true]
F = (FX(),FY(),FZ())
epsilon = zero(3,3,3)
epsilon[1,2,3] = 1
epsilon[2,3,1] = 1
epsilon[3,1,2] = 1
epsilon[3,2,1] = -1
epsilon[1,3,2] = -1
epsilon[2,1,3] = -1
A = contract(dot(epsilon,d(F,(x,y,z))),2,3)
BX = d(F[3],y)-d(F[2],z)
BY = d(F[1],z)-d(F[3],x)
BZ = d(F[2],x)-d(F[1],y)
B = (BX,BY,BZ)
-- Are A and B equal? Subtract to find out.
A-B
\end{Verbatim}

This is the result when the script runs.

$\displaystyle \begin{bmatrix}0\\0\\0\end{bmatrix}$
