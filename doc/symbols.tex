\subsection{Symbols}

Symbols are defined using an equals sign.

{\color{blue}
\begin{verbatim}
N = 212^17
\end{verbatim}
}

No result is printed when a symbol is defined.
To see the value of a symbol, just evaluate it.

{\color{blue}
\begin{verbatim}
N
\end{verbatim}
}

$\displaystyle N=3529471145760275132301897342055866171392$

\bigskip

Symbols can have more that one letter.
Everything after the first letter is displayed as a subscript.

{\color{blue}
\begin{verbatim}
NA = 6.02214 10^23
NA
\end{verbatim}
}

$\displaystyle N_A=6.02214\times10^{23}$

\bigskip

A symbol can be the name of a Greek letter.

{\color{blue}
\begin{verbatim}
xi = 1/2
xi
\end{verbatim}
}

$\displaystyle \xi=\tfrac{1}{2}$

\bigskip

Greek letters can appear in subscripts.

{\color{blue}
\begin{verbatim}
Amu = 2.0
Amu
\end{verbatim}
}

$\displaystyle A_\mu=2.0$

\bigskip

The following example shows how a symbol is scanned to find Greek letters.

{\color{blue}
\begin{verbatim}
alphamunu = 1
alphamunu
\end{verbatim}
}

$\displaystyle \alpha_{\mu\nu}=1$

\bigskip

Symbol definitions are evaluated serially until a terminal symbol is reached.
The following example sets $A=B$ followed by $B=C$.
Then when $A$ is evaluated, the result is $C$.

{\color{blue}
\begin{verbatim}
A = B
B = C
A
\end{verbatim}
}

$\displaystyle A=C$

\bigskip

Although $A=C$ is printed,
inside the program the binding of $A$ is still $B$, as can be seen with
the \verb$binding$ function.

{\color{blue}
\begin{verbatim}
binding(A)
\end{verbatim}
}

$\displaystyle B$

\bigskip

The \verb$quote$ function returns its argument unevaluated
and can be used to clear a symbol.
The following example clears $A$ so that its evaluation goes back to
being $A$ instead of $C$.

{\color{blue}
\begin{verbatim}
A = quote(A)
A
\end{verbatim}
}

$\displaystyle A$

\subsection{Functions}

The syntax for defining functions is {\it function-name} ( {\it arg-list} ) = {\it expr}
where {\it arg-list} is a comma separated list of zero to nine symbols that receive arguments.
Unlike symbol definitions, {\it expr} is not evaluated when {\it function-name} is defined.
Instead, {\it expr} is evaluated when {\it function-name} is used in a subsequent computation.
The scope of function arguments is the function definition {\it expr}.

\bigskip

The following example defines a sinc function and evaluates it at $\pi/2$.

{\color{blue}
\begin{verbatim}
f(x) = sin(x)/x
f(pi/2)
\end{verbatim}
}

$\displaystyle \frac{2}{\pi}$

\bigskip

After a user function is defined, {\it expr} can be recalled using the \verb$binding$ function.

{\color{blue}
\begin{verbatim}
binding(f)
\end{verbatim}
}

$\displaystyle \frac{\sin(x)}{x}$

\bigskip

The following example shows how \verb$eval$ is used to evaluate function arguments at specific values.

{\color{blue}
\begin{verbatim}
h(f,x,a,b) = abs(eval(f,x,a) - eval(f,x,b))
h(cos(y), y, 0, pi / 3)
\end{verbatim}
}

$\frac{1}{2}$

\bigskip

Symbols in function definition {\it expr} have global scope.
To define a local symbol, extend the argument list.
In the following example, argument \verb$y$ is used as a local symbol.
Note that function \verb$L$ is called without supplying an argument for the local symbol.

{\color{blue}
\begin{verbatim}
L(f,n,y) = eval(exp(y) / n! d(exp(-y) y^n, y, n), y, f)
L(cos(x),2)
\end{verbatim}
}

$\frac{1}{2}\cos(x)^2-2\cos(x)+1$
