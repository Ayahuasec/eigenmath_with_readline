
\subsection{Symbols}
As we saw earlier, symbols are defined using an equals sign.

\begin{Verbatim}[formatcom=\color{blue}]
N = 212^17
\end{Verbatim}

No result is printed when a symbol is defined.
To see the value of a symbol, just evaluate it.

\begin{Verbatim}[formatcom=\color{blue}]
N
\end{Verbatim}

$\displaystyle N=3529471145760275132301897342055866171392$

Symbols can have more that one letter.
Everything after the first letter is displayed as a subscript.

\begin{Verbatim}[formatcom=\color{blue}]
NA = 6.02214*10^23
NA
\end{Verbatim}

$\displaystyle N_A=6.02214\times10^{23}$

A symbol can be the name of a Greek letter.

\begin{Verbatim}[formatcom=\color{blue}]
xi = 1/2
xi
\end{Verbatim}

$\displaystyle \xi=\frac{1}{2}$

Greek letters can appear in subscripts.

\begin{Verbatim}[formatcom=\color{blue}]
Amu = 2.0
Amu
\end{Verbatim}

$\displaystyle A_\mu=2.0$

The following example shows how
Eigenmath scans the entire symbol to find Greek letters.

\begin{Verbatim}[formatcom=\color{blue}]
alphamunu = 1
alphamunu
\end{Verbatim}

$\displaystyle \alpha_{\mu\nu}=1$

When a symbolic chain is defined,
Eigenmath follows the chain as far as possible.
The following example sets $A=B$ followed by $B=C$.
Then when $A$ is evaluated, the result is $C$.

\begin{Verbatim}[formatcom=\color{blue}]
A = B
B = C
A
\end{Verbatim}

$\displaystyle A=C$

Although $A=C$ is printed,
inside the program the binding of $A$ is still $B$, as can be seen with
the $binding$ function.

\begin{Verbatim}[formatcom=\color{blue}]
binding(A)
\end{Verbatim}

$\displaystyle B$

The {\it quote} function returns its argument unevaluated
and can be used to clear a symbol.
The following example clears $A$ so that its evaluation goes back to
being $A$ instead of $C$.

\begin{Verbatim}[formatcom=\color{blue}]
A = quote(A)
A
\end{Verbatim}

$\displaystyle A$

\subsection{User-defined functions}
The following example shows
a user-defined function with a single argument.

\begin{Verbatim}[formatcom=\color{blue}]
f(x) = sin(x)/x
f(pi/2)
\end{Verbatim}

$\displaystyle \frac{2}{\pi}$

The following example defines a function with two arguments.

\begin{Verbatim}[formatcom=\color{blue}]
g(x,y) = abs(x) + abs(y)
g(1,-2)
\end{Verbatim}

$\displaystyle 3$

User-defined functions can be evaluated without an argument list.
The binding of the function name is returned when there is no
argument list.

\begin{Verbatim}[formatcom=\color{blue}]
f(x) = sin(x)/x
f
\end{Verbatim}

$\displaystyle f=\frac{\sin(x)}{x}$

Normally a function body is not evaluated when a function is defined.
However, in some cases it is required that the function body be the
result of something.
The $eval$ function is used to accomplish this.
For example, the following code causes the function body to be a sixth order Taylor series expansion of $\cos x$.

\begin{Verbatim}[formatcom=\color{blue}]
f(x) = eval(taylor(cos(x),x,6))
f
\end{Verbatim}

$\displaystyle f=-\frac{1}{720}x^6+\frac{1}{24}x^4-\frac{1}{2}x^2+1$

When a function body is evaluated the function arguments
are passed to symbol definitions.

\begin{Verbatim}[formatcom=\color{blue}]
f(x) = A + B
A = a x
B = b x
f(2)
\end{Verbatim}

$\displaystyle 2a+2b$

